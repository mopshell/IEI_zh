\section{波场外推方程}
波场外推方程是一个有关波场之导数(通常是沿深度$z$的方向)的表达式。当已知波场
及其导数时,就可利用$P(z+\Delta z)=P(z)+\Delta zdP/dz$的各种不同数值表示方法来处理外推问
题,所以真正需要的是有关$dP/dz$的表达式。求解$dP/dz$的两种理论方法是早期的变换方法
和较新的波散关系方法。

\subsection{抛物线型波动方程}
在抛物线型方程被引用于石油勘探的时期(1969年),“波动理论不起作用”这种论调
相当流行。在那个时代,石油勘探人员分析地震资料是采用射线方法,波动方程还与实际工
作无缘,唯有大学里的理论家们才会问津波动方程(事实上,波动理论对于比地震勘探尺度
大一千倍的大规模天然地震中的面波,是起过作用的)。即使是大学的研究人员,那时也未
曾完善地建立波动方程的有限差分解,计算机是计算机,解是解,所解决的多为“鼓面振
动”之类的问题,而不是求解“地层中传播的地震波”
。最早是为了提高有限差分波动模拟
的计算效率而才引入抛物线型波动方程,下面对抛物线型波动方程的介绍就是藉助于原来采
用的变换方法。

1969年以前,困难来自于有一个对当时所有地震波理论极为重要而又不恰当的假设,即
水平成层假设。射线追踪曾是摆脱该种假设限制的仅有方法,但采用射线追踪看来就得放弃
地震波形的模拟。在石油勘探中,几乎所有波动理论其本身更进一步还得受垂直入射的限
制。成功地克服困难的途径就在于将垂直入射情形加以推广,沿垂直入射方向周围允许有微
小角度变化范围,放弃许多熟悉但很麻烦的地震理论就达到了这个目的。

垂直下行平面波在数学上以下述方程表示
\begin{equation}
P(t,x,z)=P_0e^{-i\omega (t-z/v)}
\label{eq:ex2.1.1}
\end{equation}
式中,$P_0$纯为常数。将$P_0$用某种不是严格恒定而是缓慢变化的函数$Q(x,z)$来代替,即可
模拟偏离垂直入射的微小角度改变,即
\begin{equation}
P(t,x,z)=Q(x,z)e^{-i\omega (t-z/v)}
\label{eq:ex2.1.2}
\end{equation}
将式\ref{eq:ex2.1.2}代入标量波动方程$P_{xx}+P_{zz}=P_{tt}/v^2$,得
$\frac{\partial^2 Q }{\partial x^2}+(\frac{i\omega}{v}+\frac{\partial}{\partial z})^2Q
=-\frac{\omega^2}{v^2}Q$
即
\begin{equation}
\frac{\partial^2 Q }{\partial x^2}+\frac{2i\omega}{v}\frac{\partial Q}{\partial z}+\frac{\partial^2 Q}
{\partial z^2}=0
\label{eq:ex2.1.3}
\end{equation}
导出此式时,未作任何假设,仅仅是将波动方程用$Q(x,z)$重新加以表示而已。为使波场接近于平面波,$Q(x,z)$
必须接近于一常数。适宜的假设应是$Q$沿深度的最高阶导数、即$Q_{xx}$可忽略不计(首次引入这个假设时,曾引起一些争论),这就使我们得出抛物线型波动方程
\begin{equation}
\frac{\partial Q }{\partial z}=-\frac{v}{2i\omega}\frac{\partial^2 Q }{\partial x^2}
\label{eq:ex2.1.4}
\end{equation}
首次建立这个方程用于地震学的时候,当时认为式\ref{eq:ex2.1.4}的最重要性质是这样一
点:对于接近于沿垂直方向传播之平面波的,一种波场而言,沿$x$轴方向的二阶导数应很小,
因而沿$z$轴方向的导数应很小。所以,应用有限差分方法时将允许采用非常大的步长$\Delta z$,从
而能使处理的模型更像是地层模型而不大像是鼓面。

随后,很快就弄明白了,抛物线型波动方程也正是地震成像方法所需要的那种方程,即
它是一种波场外推方程。

妙极了,式\ref{eq:ex2.1.4}就是量子力学中的Schroedinger方程的形式。

这种办法、即变换方法曾经是而且现在也是非常有用的。不过它很快就为波散方程处理
方法所取代,这是获得以较宽角度进行波场外推的方程的途径。
\subsection{Muir平方根展开方法}
在我们采用较新的求解波场外推算子的方法时,要探索平方根波散关系的各种不同近
似,然后,将近似波散关系反变换为一个偏微分方程。自从我的《地球物理数据处理基础》
一书写成以来,波散关系处理方法已经有了显著进展,这得大大感谢Francis
Muir。

将平面波$exp(-i\omega t+ik_xx+ik_zz)$代入二维标量波动方程,得出波散关系
\begin{equation}
k_x^2+k_z^2=\frac{\omega^2}{v^2}
\label{eq:ex2.1.5}
\end{equation}
求解$k_z$,选择正平方根(选择下行波时即如此)
\begin{subequations}
\begin{equation}
k_z=\frac{\omega}{v}\sqrt{1-\frac{v^2k_x^2}{\omega^2}}
\label{eq:ex2.1.6a}
\end{equation}
为沿$z$轴进行反变换,我们仅需理解相应于反变换所得最终表达式是一个波
场外推算子,即
\begin{equation}
\frac{\partial P}{\partial z}=i\frac{\omega}{v}\sqrt{1-\frac{v^2k_x^2}{\omega^2}}P
\label{eq:ex2.1.6b}
\end{equation}

\end{subequations}
