\section{分裂法与全分离法}
\label{sec:2.4}
通常同时起作用的两个过程A与B也许是或者
也许不是相互联系的。它们相互独立的这种情形, 称作全分离(full separation)
。在这种情形下,就概念和就计算而言,想像A过程在B过程开始之
前正趋近于结束,这往往是很有用的。在两种过程
彼此有相互联系的场合,有可能是允许A短暂起作
用,然后转换为B起作用,并如此交替作用下去,
这种交替起作用的办法称作分裂法(splitting)。

\subsection{热流方程}
\label{sec:2.4.1}
绕射方程或偏移方程可称作“波阵面恢复”方程,它把初始条件或透镜项可能引起的波阵面的任
何横向突变一起加以平滑恢复原状。
15°偏移方程具有有与热流方程相同的数学形式。不过热
流方程全是实数,而且它的物理性态更容易理解,这点值得多说两句:(l)x方向的热流
$H_x$等于温度的负梯度$-\partial T/\partial x$如乘以热传导率$\sigma$。(2)温度降低$-\partial T/\partial t$是同热流发散量$\partial H_x/\partial x$
除以热容量c成比例。将上述二者结合起来并由一维情形推广至二维情形,取$\sigma$为常数及
$c=l$,得出方程:
\begin{equation}
\frac{\partial T}{\partial t}=\sigma[\frac{\partial^2}{\partial x^2}+\frac{\partial^2}{\partial y^2}]T
\label{eq:ex2.4.1}
\end{equation}

\subsection{分裂法}
\label{sec:2.4.2}
应用于热流方程数值解法的分裂法是用两个微分方程代替热流方程,按交替的时间步长
应用其中每个方程
\begin{subequations}
\begin{equation}
\frac{\partial T}{\partial t}=2\sigma\frac{\partial^2 T}{\partial x^2} \quad (all\quad y)
\label{eq:ex2.4.2a}
\end{equation}
\begin{equation}
\frac{\partial T}{\partial t}=2\sigma\frac{\partial^2 T}{\partial y^2} \quad (all\quad x)
\label{eq:ex2.4.2b}
\end{equation}
\label{eq:ex2.4.2}
\end{subequations}

式\ref{eq:ex2.4.2a}中,对于x方向的热流其热传导率$\sigma$业已增大两倍,而对$y$方向的热流则已取$\sigma$
为零;在式\ref{eq:ex2.4.2b}中,则情形反之。在时间的奇数时刻,热量按式\ref{eq:ex2.4.2a}的关系流
动;在时间的偶数时刻则按式\ref{eq:ex2.4.2b}的关系流动。这种轮流交替采用式\ref{eq:ex2.4.2a}与
\ref{eq:ex2.4.2b}所得的解在数学上可证明是收敛于式\ref{eq:ex2.4.1}的解,其误差为$\Delta t$数量级,因此
汾趋于零时,误差亦趋于零。高维隐式方法的不可行性是促使采用分离法的原因(参阅\ref{sec:2.2}
节结尾部分)。

\subsection{全分离法}
\label{sec:2.4.3}
