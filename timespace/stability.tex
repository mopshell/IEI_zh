\section{稳定性简介}
\label{sec:2.8}

经验表明,一旦你认识到方法应用显著偏离了教科书所述情况,稳定性雜要比精确度更
为受到关心。有没有稳定性将决定预计目标是否能完全达到,而精度则仅决定达到目标所需
付出的计算代价。我们在本节将考虑具有实热传导系数和虚热传导系数的热流方程。由于后
种情形相应于地震偏移,所以这两种情形为稳定性分析提供了有益的背景。

稳定性分析的基本方法系以傅氏变换为基础,更简单地说,我们要考察的是单个正弦形
或复指数形的试验解。如果一种方法对任何频率变得不稳定,那么,它对任何实际情形也将是
不稳定的,因为实际函数只不过是所有频率的合成结果。现在就从下述正弦函数开始讨论:
\begin{equation}
P(x)=P_0e^{ikx}
\label{eq:ex2.8.1}
\end{equation}
其二阶导数为
\begin{equation}
\frac{\partial^2 P }{\partial x^2} =-k^2P
\label{eq:ex2.8.2}
\end{equation}
用类似于二阶差分算子的一个表达式来定义$\hat{k}$
\begin{subequations}
  \begin{equation}
  \frac{\delta^2 P }{\delta x^2}=\frac{P(x+\Delta x)-2P(x)+P(x-\Delta x)}{\Delta x^2}
  \label{eq:ex2.8.3a}
  \end{equation}
  \begin{equation}
  =-\hat{k}^2P
  \label{eq:ex2.8.3b}
  \end{equation}
\label{eq:ex2.8.3}
\end{subequations}
理想上,$\hat{k}$应该等于$k$。将复指数\ref{eq:ex2.8.1}代入式\ref{eq:ex2.8.3a},得出关于$\hat{k}$的表达式:
\begin{subequations}
  \begin{equation}
  -\hat{k}^2P=\frac{P_0}{\delta x^2}[e^{ik(x+\Delta x)}-2e^{ikx}+e^{ik(x-\Delta x)}]
  \label{eq:ex2.8.4a}
  \end{equation}
  \begin{equation}
  (-\hat{k}\Delta x)^2=2[1-cos(k\Delta x)]
  \label{eq:ex2.8.4b}
  \end{equation}
作出式\ref{eq:ex2.8.4b})的图形或者它的平方根的图形,是一件轻而易举的事。利用三角学中的半角
恒等式,可将\ref{eq:ex2.8.4b}的平方根表示为
\begin{equation}
\hat{k}\Delta x=2sin\frac{k\Delta x}{2}
\label{eq:ex2.8.4c}
\end{equation}
\label{eq:ex2.8.4}
\end{subequations}
作级数展开后表明,$\hat{k}$与$k$在低空间频率时符合良好。在$k\Delta x=\pi$的关系所定义的Nyquist频
率时,值$\hat{k}\Delta x=2$只粗略地近似于$\pi$。
与离散域上的任何傅氏变换一样,超过Nyquist频率
时,$\hat{k}$是$k$的一个周期函数。虽然$k$的范围是从负无限大至正无限大,$\hat{k}^2$却压缩成从零至四的
范围。由于不稳定性往往是在值域范围的一个端点上开始,所以变化范围的极限是很重要
的。

\subsection{显式热流方程}
\label{sec:2.8.1}

现在就从热流方程和空间域傅氏变换开始讨论。$\partial^2/\partial x^2$直接变为$-k^2$,因而
\begin{equation}
\frac{\partial q}{\partial t}=-\frac{\sigma}{c}k^2q
\label{eq:ex2.8.5}
\end{equation}
时间域显式有限差分得出的方程在形式上同通货膨胀方程完全相同:
\begin{subequations}
  \begin{equation}
  \frac{q_{t+1}-q_t}{\Delta t}=-\frac{\sigma}{c}k^2q_t
  \label{eq:ex2.8.6a}
  \end{equation}
  \begin{equation}
  q_{t+1}=(1-\frac{\sigma\Delta t}{c}k^2)q_t
  \label{eq:ex2.8.6b}
  \end{equation}
\label{eq:ex2.8.6}
\end{subequations}
为保证稳定性,$q_{t+1}$的量值应当小于或者等于$q_t$的量值,这就要求括号内的因子具有小于或
等于一的量值。危险的情况发生在因子甚小于$-1$
之时,当$k^2>2c/(\sigma\Delta t)$时就要出现不稳
定性,这意味着高频分量是随时间而发散的,在时间坐标轴上实现显示有限差分给空间坐标
轴上的短波长招致了灾难性后果。令人惊异的是,只要对空间坐标轴进行的差分足够粗略就
能够补救这种灾难!傅氏变换域内的二阶空间导数为$-k^2$,当$x$坐标轴离散化时,它变为
$-\hat{k}^2$。所以,使式\ref{eq:ex2.8.5}与\ref{eq:ex2.8.6}离散化,只不过是用$\hat{k}$代替$k$而已。式\ref{eq:ex2.8.4c}表
明,在Nyquist频率$k\Delta x=\pi$时,$\hat{k}^2$有一个上限$\hat{k}^2=4/\Delta x^2$。最后,如果
\begin{equation}
\hat{k}^2=\frac{4}{\Delta x^2}\leq\frac{2c}{\sigma\Delta t}
\label{eq:ex2.8.7}
\end{equation}
式\ref{eq:ex2.8.6b}中的因子将小于一,从而计算过程就具有稳定性。显然,时间采样比空间采样
稠密可防止不稳定性。不过,当热传导系数$\sigma(x)$取值范围很广时,这么一种解决办法就
代价太大了。对于一维空间问题,有种很容易逃避的办法,就是采用隐式差分方法;对于高
维空间问题,则必须采用显式差分方法。

\subsection{显式15度偏移方程}
\label{sec:2.8.2}

在\ref{sec:2.1}节中我们已经知道,除了必须用纯虚数$i$代替热传导系数$\sigma$之外,延迟的15度波场外
推方程很像热流方程,放大因子(式\ref{eq:ex2.8.6b}括号中之因子的大小)现在是实部与虚部之平方
和的平方根。既然实部已经是$1$,则放大因子在$k^2$的所有非零值情形下均超过$1$。随着倾斜平
面波的増大,无疑要显现出不稳定性,倾角越大,增长越快,而且增大$x$轴的采样间隔也解
决不了这个问题。

\subsection{隐式方程}
\label{sec:2.8.3}
以前已说过,通货膨胀方程
\begin{equation}
q_{t+1}-q_t=rq_t
\label{eq:ex2.8.8}
\end{equation}
就是微分方程$dq/dt\approx q$的一种简单显式有限差分。还知道,Crank-Nicolson形式
\begin{subequations}
  \begin{equation}
  \frac{q_{t+1}-q_t}{\Delta t}=r\frac{q_{t+1}+q_t}{2}
  \label{eq:ex2.8.9a}
  \end{equation}
可以对微分方程给出较好的近似,这种形式可重写为
  \begin{equation}
  (1-\frac{r}{2})q_{t+1}=(1+\frac{r}{2})q_t
  \label{eq:ex2.8.9b}
  \end{equation}
或者
\begin{equation}
\frac{q_{t+1}}{q_{t}}=\frac{1+r/2}{1-r/2}
\label{eq:ex2.8.9c}
\end{equation}
\label{eq:ex2.8.9}
\end{subequations}
对于$r$的所有负值、甚至当$r$等于负无限大时,式\ref{eq:ex2.8.9c}的放大因子的量值都小于$1$。记
住热流方程相应于
\begin{equation}
r=-\frac{\sigma\Delta t}{c}k^2
\label{eq:ex2.8.10}
\end{equation}
其中,$k$为空间频率。既然式\ref{eq:ex2.8.9c}适用于$r$的所有负值情形,则采用隐式时间差分的热
流方程就适用于所有空间频率$k$。不论空间坐标轴是否离散采样(离散采样则$k\rightarrow\hat{k}$)而且无
论$\Delta t$与$\Delta x$的大小如何,热流方程都是稳定的。此外,15度波场外推方程也是无条件稳定的。
令式\ref{eq:ex2.8.4c}中的$r$为纯虚数就可导出这个结论,这时式\ref{eq:ex2.8.9c}的放大因子所取形式
为某种复数$l+r/2$被其复共轭所除。以极坐标形式表示复数时,这样一个数具有严格等于$1$的
量值就更清楚了。因此说,它是无条件稳定的。

关于这点,多作点历史脚注看来是必要的。当初引入有限差分偏移时,由于都不熟悉
其理论假设,曾引起许多非议。尽管有非议,有限差分偏移还是很快流行起来了,我想,其
所以能流行的原因就在于:同其他的时间域方法比较起来,它是一种优美的数据运算。更具
体地说,由于式\ref{eq:ex2.8.9c}的大小严格等于$1$,于是输出就具有与输入是相同的$(\omega,k)$
谱。可能都有这样的经验教训:任何作用于数据的矬理过程都应该尽可能少地影响数据。

\subsection{蛙跃式方程}
\label{sec:2.8.4}

人们会回想到,蛙跃式有限差分法要求在两个时间步长上表示时间导数,这样作可使差
分算子的中心保持在同样的位置上。就经过空间域傅氏变换的热流方程而言
\begin{equation}
\frac{q_{t+1}-q_t}{2\Delta t}=-\frac{\sigma}{c}k^2q_t
\label{eq:ex2.8.11}
\end{equation}
要分析这个方程还真有点讨厌,因为它涉及三个时间$t-1$,$t$和$t+1$,并且要求采用稍微更困
难一些的解析方法。因此,首先阐明结论看来足值得的。就热流方程而言,结论就是:解总楚
发散的。就波场外推方程而言,所得结论非常之有用:倘若满足某种关于网格大小的限制,
即$\Delta z$必须小于某个因子乘以$\Delta x^2$,则解恒为稳定。对于一维空间,这种结论并不令人惊奇
(这神情形下,隐式方法似乎是颇理想的),但是对于高维空间,诸如在所谓三维地震勘探
那种情形下,我们或许得感谢蛙跃法。

要分析像式\ref{eq:ex2.8.11}那样的范围涉及三个或三个以上时间步长的方程,最好的途径就
是利用$Z$变换滤波分析。使之转换为$Z$变换滤波问题后,式\ref{eq:ex2.8.11}所提出的问题就变成了
该滤波器在单位圆之内(或之外)是否有零点的问题了。在\ref{sec:4.6}节内将阐述$Z$变换稳定性分析
方法,对的所有可能的数值,都有必要进行这样的分析。结论就是:如$k^2$的范围是从零至
无限大,则总有麻烦存在。不过,对于波场外推方程,利甩某种网格大小的限制,是可以避
免不稳定性的,因为$(\hat{k}\Delta x)^2$介于零与四之间。

\subsection{三对角线方程的解法}
\label{sec:2.8.5}

三对角线算法对所有正定矩阵都是稳定的,如果你的三对角线解法有任何问题,那就应
怀疑你的问题公式是否成立;在看来似乎是要求用零来除的地方,你在应用中采取了什么办
法?
