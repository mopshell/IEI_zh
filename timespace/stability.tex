\section{稳定性简介}
\label{sec:2.8}

经验表明,一旦你认识到方法应用显著偏离了教科书所述情况,稳定性雜要比精确度更
为受到关心。有没有稳定性将决定预计目标是否能完全达到,而精度则仅决定达到目标所需
付出的计算代价。我们在本节将考虑具有实热传导系数和虚热传导系数的热流方程。由于后
种情形相应于地震偏移,所以这两种情形为稳定性分析提供了有益的背景。

稳定性分析的基本方法系以傅氏变换为基础,更简单地说,我们要考察的是单个正弦形
或复指数形的试验解。如果一种方法对任何频率变得不稳定,那么,它对任何实际情形也将是
不稳定的,因为实际函数只不过是所有频率的合成结果。现在就从下述正弦函数开始讨论:
\begin{equation}
P(x)=P_0e^{ikx}
\label{eq:ex2.8.1}
\end{equation}
其二阶导数为
\begin{equation}
\frac{\partial^2 P }{\partial x^2} =-k^2P
\label{eq:ex2.8.2}
\end{equation}
用类似于二阶差分算子的一个表达式来定义$\hat{k}$
\begin{subequations}
  \begin{equation}
  \frac{\delta^2 P }{\delta x^2}=\frac{P(x+\Delta x)-2P(x)+P(x-\Delta x)}{\Delta x^2}
  \label{eq:ex2.8.3a}
  \end{equation}
  \begin{equation}
  =-\hat{k}^2P
  \label{eq:ex2.8.3b}
  \end{equation}
\label{eq:ex2.8.3}
\end{subequations}
