\section{递归倾角滤波}
\label{sec:2.5}
递归滤波是一种将滤波输出反馈再作为输入的滤波形式。这种滤波只需微少计算时间即
可得出长的脉冲响应,在滑动平均的计算中特别有用。滑动平均可以实现频率域内的低通滤
波作用,但是一般最好还是避免进行空间变换。物理空间是比较方便的,它容许系数可变,
而且它允许更为灵活地处理边界问题。无论在空间域或时间域,地球物理数据组很少有在长
距离上呈平稳状态的,所以,递归滤波在统计估计问题中特别有用。

大多数滤波的目的都是想使被强同相轴所掩盖的重要的微弱同相轴有可能被观测到。一
维滤波仅靠对频率分量迸行选择或抑制才能作到这点;在二维情形下,则有可能采取一种不
同的准则,即倾角选择作用。

倾角滤波是地球物理学家长期以来感兴趣的一种处理(Embree, Burg及Backus,
1963)。陡倾角往往是地滚波干扰,水平倾角也可以是干扰。例如,弱断层绕射具有有价值
的信息,可是由于平缓地层的存在占主导优势,它们往往可能看不清楚。

要作普通的倾角滤波运算(扇形滤波),你只需将数据变换至($(\omega,k)$域,乘以任何
希望的与$k/\omega$有关之函数,然后再变换回去。扇形滤波就是这样对$k/\omega$倾角空间内的滤波响
应加以完全控制,而控制递归倾角滤波就不如此容易了,它们像扇形滤波一样可满足相同的
一般需要,而且还能提供下列的额外好处:
\begin{enumerate}
\item 时间可变性与空间可变性;
\item
  具有时间因果性;
\item
  易于实现;
\item
  计算时间比在$(\omega,k)$域内实现节省很多。
\end{enumerate}
时间因果性这种性质为进行数据记录提烘了一种有意义的可能,即可将水层速度截阻滤
波作用装进现代高密度海上电缆的记录装置内去进行,实现软件硬化。

\subsection{递归倾角滤波定义}
\label{sec:2.5.1}
令$P$表示原始数据,$Q$表示经过滤波处理之后的数据。当地震资料是准单频情形时,用
下列空间频率滤波可完成倾角滤波,其中,$\alpha$为可调截频参量:

\begin{table}
\centering
\begin{tabular}{|c|c|}
\hline
\multicolumn{2}{|c|}{单频资料情形下的倾角滤波($\omega\approx$常数)}\\ \hline
低通&$Q=\frac{a}{a+k^2}P$\\ \hline
高通&$Q=\frac{k^2}{a+k^2}P$ \\ \hline
\end{tabular}
\end{table}

要在空间域内应用这些滤波,仅需将$k^2$解释为三对角线矩阵$\mathbf{T}$,其主对角线上的元素为
$(-1,2,-1)$。具体说,对于低通滤波,需要求解下述三对角线联立方程组
\begin{equation}
(\alpha\mathbf{I}+\mathbf{T})\mathbf{q}=\alpha\mathbf{p}
\label{eq:ex2.5.1}
\end{equation}
式中,$\mathbf{q}$与$\mathbf{p}$为列向量,其元素代表$x$轴上的不同位置(译注:式中的$\mathbf{I}$为单位矩阵)。以前求
解热流方程时,就已经用过这种矩阵。为使滤波是空间可变的,可取参量$\alpha$使与$x$有关,从而可
用一种任意的对角线矩阵来代替$\alpha\mathbf{I}$。究竟是在频率$\omega$域内还是在时间$t$域内来表示$\mathbf{p}$与$\mathbf{q}$,没什
么关系。

现在从狭频带资料转而注意具有较宽一点的'频谱之资料,这时可参考下述滤波:

\begin{table}
\centering
\begin{tabular}{|c|c|}
\hline
\multicolumn{2}{|c|}{中等频带宽度($\Delta\omega$)资料情形下的倾角滤波} \\ \hline
低通&$Q=\frac{a}{a+\frac{k^2}{-i\omega}}P$\\ \hline
高通&$Q=\frac{\frac{k^2}{-i\omega}}{a+\frac{k^2}{-i\omega}}P$ \\ \hline
\end{tabular}
%\caption{2.5.2}
%\label{tab:2.5.2}
\end{table}

这些滤波自然可以应用于任何频带宽度的资料。不过,只有在适当的中等频带宽度范围内才
能恰如其份地被称作“倾角滤波”。
