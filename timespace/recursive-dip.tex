\section{递归倾角滤波}
\label{sec:2.5}
递归滤波是一种将滤波输出反馈再作为输入的滤波形式。这种滤波只需微少计算时间即
可得出长的脉冲响应,在滑动平均的计算中特别有用。滑动平均可以实现频率域内的低通滤
波作用,但是一般最好还是避免进行空间变换。物理空间是比较方便的,它容许系数可变,
而且它允许更为灵活地处理边界问题。无论在空间域或时间域,地球物理数据组很少有在长
距离上呈平稳状态的,所以,递归滤波在统计估计问题中特别有用。

大多数滤波的目的都是想使被强同相轴所掩盖的重要的微弱同相轴有可能被观测到。一
维滤波仅靠对频率分量迸行选择或抑制才能作到这点;在二维情形下,则有可能采取一种不
同的准则,即倾角选择作用。

倾角滤波是地球物理学家长期以来感兴趣的一种处理(Embree, Burg及Backus,
1963)。陡倾角往往是地滚波干扰,水平倾角也可以是干扰。例如,弱断层绕射具有有价值
的信息,可是由于平缓地层的存在占主导优势,它们往往可能看不清楚。

要作普通的倾角滤波运算(扇形滤波),你只需将数据变换至($(\omega,k)$域,乘以任何
希望的与$k/\omega$有关之函数,然后再变换回去。扇形滤波就是这样对$k/\omega$倾角空间内的滤波响
应加以完全控制,而控制递归倾角滤波就不如此容易了,它们像扇形滤波一样可满足相同的
一般需要,而且还能提供下列的额外好处:
\begin{enumerate}
\item 时间可变性与空间可变性;
\item
  具有时间因果性;
\item
  易于实现;
\item
  计算时间比在$(\omega,k)$域内实现节省很多。
\end{enumerate}
时间因果性这种性质为进行数据记录提烘了一种有意义的可能,即可将水层速度截阻滤
波作用装进现代高密度海上电缆的记录装置内去进行,实现软件硬化。

\subsection{递归倾角滤波定义}
\label{sec:2.5.1}
令$P$表示原始数据,$Q$表示经过滤波处理之后的数据。当地震资料是准单频情形时,用
下列空间频率滤波可完成倾角滤波,其中,$\alpha$为可调截频参量:
