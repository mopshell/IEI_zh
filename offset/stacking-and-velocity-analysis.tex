\section{叠加与速度分析}
\label{sec:3.5}

按炮检距沿双曲线迸行叠加,可能是地震勘探中最重要的计算机处理过程了。比它偏移
更为重要,因为它使数据库从$(s, g, t)$空间的一个体积简化为$(y, t)$空间的一个平面。
地震资料解释人员现在还很少有人采用计算机化的电影式地震资料,所以大多数解释人员在
他们能解释分析之前都必须使地震资料经过叠加处理。偏移只不过就是把资料从一个平面转
换至另一个平面。此外,偏移有个缺点:有时它把近地表速度横向变化和多次反射造成的混
乱也混杂在一起了。叠加处理也可能混杂有这类影响,不过,在条件恶劣的地区,资料要不
经过叠加处理,那就别想看出什么东西。除前面列举的各点之外,叠加处理还有个附带的产
品,那就是据此可以估计岩石速度。

从历史上说,曾经采甩射线法来完成叠加处理,而且现在仍然几乎是唯一地按这种方式来
处理。而另一方面,偏移则总是采用波动方程方法来完成,那就是说,以Fourier变换或有
限差分方法来实现。偏移与叠加二者均属涊曲线识别处理过程,偏移时采用波动方程方法有
许多好处,这些好处难道不能同等地适用于益加吗?看起来似乎是应该如此,但是现在的勘
探实践还没有怔实这点,其原因还不太清楚。本章以后各节内容实际上是属于专题研究性
质,其标题不紡戏称为《不久将要建立的理论》。有关速度估计的更高级概念在\ref{sec:5.0}节至\ref{sec:5.4}
节讨论。波动方程叠加与速度测定方法均是独创性方法,它们也许还缉有经受过令人满意的
检验,或者它们也许还只是不完善地被汇总在一起的。给读者留有想像的余地,时间将会告
诉我们究竟如何。

为什么许多这种理论并没在常规勘探工作中应用?可能的一种原因是:为消除冗余信息
而进行叠加这种论点大概更适合于作为一个统计问题来处理,而不能作为一种物理问题来处
理。为考虑到这类随机偶然性,我得多用一点笔墨来讨论“波动方程时差校正”,这是一种
将统计分析推迟到向下延拓处理之后再迸行的方法。另一种可能原因则是:比起波动方程
来,用射线法解决:排列末端数据丢失和雄列长度范围内出现空间假频的问题,大概要更为灵
活。对于这类随机偶然性质,我特意列了简短一小节来讨论数据恢复问题。不论是哪种情
况,本章论述的数据处理办法想来都应有所裨益。

\subsection{正常时差校正(NMO)}
\label{sec:3.5.1}

正常时差校正(NMO)就是对时间轴进行某种拉伸,使所有地震记录看来像是零炮检
距地震记录。\ref{sec:3.0.1}节内曾首先讨论过NMO,在那种最简单的形式中,NMO是以毕达哥拉斯
关系$t_{nmo}^2=t^2-x^2/v^2$为基础的。在地层的速度为恒定时,正常时差校正要取双曲线族的渐
近线并沿它移动直至与$t=0$重合,这就是在时间轴上舍去初至之前的任何采样值,然后再将
地震记录的其余部分拉平。接近初至之处,拉伸影响最为显著,在以后的各时间上则逐次减
弱。在图\ref{fig:vdmo/cmpnmo}所示的正常时差校正例子中,你会注意到有由拉伸而形成的许多低频成分。
\begin{figure}[H]
\centering
\includegraphics[width=0.65\textwidth]{vdmo/cmpnmo}
\caption[cmpnmo]{左图所示为墨西哥海湾地区的共中心点道集,经正常时差校正后如右图所示}
\label{fig:vdmo/cmpnmo}
\end{figure}

能够作正常时差校正的是共炮点道集或共中心点道集。应用于共炮点道集的正常时差校
正使该道集成为类似于零炮检距时间剖面的一小部分,这时地质构造突出地被展现出来。对共
中心点道集进行正常时差校正,是确定地层的速度与深度之函数关系的主要手段,这是因为共
中心点道集对地层之倾角是不灵敏的。

从数学上说,有关正常时差NMO的变换是一种线性运算。事情看来似乎有些矛盾,一种
非均匀的时间坐标拉伸运算竟是某种线性运算,可是坐标拉伸确实是满足与线性性质有关的
数学条件。不要把流行的线性条件混淆为不太常见的时间不变量条件。线性性质仅要求;对
于将原始数据P分成几部分(例如分成与$P_1$与$P_2$)的任何分解来说,经过正常时差校正的几
部分之和应等于和之正常时差校正。分解的例子包括:(1)分成较早各时间和较晚各时间;
(2)分成偶数时间点和奇数时间点;(3)分成高频和低频;(4)分成大信号值与小信号值。

为把正常时差校正想像成是一种线性算子,试将地震记录考虑为某种向量。NMO算子
类似于一个对角线矩阵,但是沿矩阵对角线分布的是内插滤波因子,而且各内插滤波因子均
有偏离对角线的相移以形成所期望的时延。

\subsection{常规速度分析}
\label{sec:3.5.2}

常规速度分析要利用一系列试验速度,每种试验速度取为深度的恒定函数并用它对数据
进朽时差校正。图\ref{fig:vdmo/cmphale}的左图为用一恒定速度对图\ref{fig:vdmo/cmpnmo}
所示共中心点道集进行时差校正之后的结果。注意,该道集中部的各同相额均将近拉平,而较小时间上的同相轴则校正不足,
较大时间上的同相轴则过校正。这种现象很典型,因为正常时差校正量的变化是与速度
呈反比关系(根据毕达哥拉斯关系即可知),而地层的速度在通常情形下正是随深度而增大
的。利用遍及CDP道集各炮检距进行求和的结果作为时差校正速度是否良好拟合于地层速度
的一神测度,大概说来,速度拟合程度越佳,求和值就将越大,用许多种速度重复进行这种
过程,按求和值的幅度绘制等值线图,将它显示为时间与速度的函数,这就是速度谱,如图
\ref{fig:vdmo/cmphale}的右图所示。

\begin{figure}[H]
\centering
\includegraphics[width=0.65\textwidth]{vdmo/cmphale}
\caption[cmphale]{按恒定速度进行正常时差校正及速度分析结果}
\label{fig:vdmo/cmphale}
\end{figure}

在实际处理时,求和之前可
以采用一些额外的修饰处理步骤。可以按记录道的功率和按它
们的谱(见\ref{sec:5.5}节有关反褶积处理的论述)对各记录道作道间均
衡处理(使它们比例相等)。类似还有,可以将求和所得幅值加
以平滑和归一化(见Taner与 Koehler(l969)的论文)。还可
以如下节所述将数据加以编排和加权。

得出最佳叠加效果的速度是 反射面以上各地层之速度的某一
种平均值,这种平均值的精确定义推迟到\ref{sec:5.4}节再去讨论。

\subsection{切除与加权}
\label{sec:3.5.3}

定义切除函数是常规赴理中
的重要一部分。切除函数就是用 于压制掉数据中某些不希望要的
部分时所使用的一种加权函数。图\ref{fig:vdmo/mute}所示是经过切除处理之野外剖面的例子。加权与切
除对叠加的质量有重大影晌。因此,有许多试验和理论讨论实际上都是以它们作为研究主
题,这是毫不奇怪的。

\begin{figure}[H]
\centering
\includegraphics[width=0.65\textwidth]{vdmo/mute}
\caption[mute]{左图是Alberta地区的陆地勉震记录,右图是该记录经过切除处理后
的结果,地滚波(记录中部)与首波(第一个波至)均已消除}
\label{fig:vdmo/mute}
\end{figure}

切除函数往往是以r为变量的一维函数,其中\footnote{h为炮检距,t为时间.------译者},$r=h/t$。
既切除大r值时的数据也切除小r值时的数据,其原因如下所述。

在小r值时,炮点附近仍然可发现有能量存在,诸如水或泥块坠落所形成的影响,或者
低速地滚波所形成的干扰能量。

r值大时,则存在与初至有关的问题。在这种情形下,正常时差拉伸最大而且对速度极
敏感。这初至往往称作首波或折射波。从野外观测看,首波就是一种其旅行时间表现为距离
之线性函数的波。就理论上考虑,针对分层介质很容易对首波作出解释。首波具有沿一地层
边界面作水平传播的射线。在实际现象中,首波可以弱于或者是强于反射波,首波强于反射
波可以用反射波作三维分布而首波仅作二维分布这个事实来解释。

可以把切除楚理看作是用零值迸行加权。为产生最有利的叠加效果,可以选择更普遍性
的加权办法。完善的分析肯定应将干扰噪音和截断影响考虑在内,我们权且作一个简化分析
吧,它可导出最基本的加权函数。

一般我们是沿双曲线遍及所有炮检距进行求积。考虑三维问题时则与此不同,你这时实
际是希望遍及一旋转双曲面进行求积,假设该双曲面是呈径向对称的,按炮检距h的大小对
被积函数进行加权,就使得通常的线性积分能够模拟出遍及旋转双曲面的积分结果
\footnote{这是必须迸行加权处理的第一个理由.所谓沿双曲线的积分,实即正常时差校正并叠加;所罚沿旋转双曲面求
积,就是考虑球面扩散影响时的叠加。因此,必须按炮检距的大小,成比例地对叠加道(所谓被积函数)进行加权,校正
该种影响,这时的叠加处理(所谓的线性积分)结果,才能消除三维情形下的能量扩散。直接了当地说,进行加权处理的
第一个理由就是必须对波阵面球面扩散影响进行适当的补偿。}。叠加
前要按炮检距h的大小对数据进行按比例放大,还有第二个原因,那就是在零炮检距附近能
获得的速度信息不多,这些记录道的时差很小而在远炮检距时则可获得很多速度信息,这些
记录道的比值比较大
\footnote{这是必须进行加权处理的第二个理由。浅显地说,近炮点记录道时差小,因而速度分辨率低;远炮点记录道时差
大,速度分辨率高。但是近炮点记录道振幅强,而远炮点记录道因能量衰减而振幅弱,采用以叠加方法为基础的速度分析
时,将因此而主要是反映近炮点记录道的作周和影响,降低了速度分析的精度。因此必须按炮检距的大小而成比例地加权
放大记录道辐值,使得近炮点与远炮点记录道的振幅在叠加中发挥同等作用。这种加权处理,对远度谱而言,可提高速度
分辨率;对叠加处理来说,可提高叠加效果和质量。----译者}。

\subsection{NMO(正常时至校正)方程}
\label{sec:3.5.4}

在任一已知记录资料的深度范围之内,地层速度变化范围可达2倍左右,是很典型的情
形。这样一来,对于以毕达哥拉斯关系为基础的分析方法就得需要重新加以考虑砑究了。在
实际处理中,总是采取在毕达哥拉斯关系中插入一个随时间而变的速度这种办法来处理随深
度而变的速度(有关的经典参考文献见Taner与Koehler[l969]的论文,它包括有许多颇有
助益的细节讨论)。尽管计算正确的非双曲线时差校正并不困难,很多还是采用这种近似。
让我们看一看倒底是如何在数学上将速度函数$v(z)$同正常时差联系起来的。设共中心点道
集之一记为$P(h,t)$,正常时差校正将该CMP道集转换为某一地层模型$Q(h,z)$
\begin{equation}
Q(h,z)=earth(z)\times const(h)
\label{eq:ex3.5.1}
\end{equation}
实际上,$Q(h,z)$终究还不是$h$的恒定函数,它还只是我们所欲达到的目标。

可以把正常时差校正处理看成是一种简单的复制过程。这点在概念上很容易想像,它就
是把$(h,t)$平面上的每一个点复制拷贝到$(h,z)$平面上的适当位置。可将这一类拷贝过程记为
\begin{equation}
Q[h,z(h,t)]=P(h,t)
\label{eq:ex3.5.2}
\end{equation}
这么拷贝时,必须小心避免在$(h,z)$平面上留下一些空洞
\footnote{正常时差校正值需为时间采样间隔$\Delta t$的整涪数,但应校正的值不一定诒为其整数倍。在校正处理过程中,
超过整数倍之余值累积到等于一个$\Delta t$值时,须进行内插。所谓“空洞”,即应内插之处。----译者}。
最好是对$(h,z)$输出平面
内的每一个点迸行扫描,从而从$(h,t)$平面内找出其在$(h,z)$平面内产生该空洞的根
源。利用某种动校正时间表$t(h,z)$,可采取下列拷贝处理过程,对数据加以时差校正
\begin{equation}
Q[h,z(h,t)]=P[h,t(h,z)]
\label{eq:ex3.5.3}
\end{equation}

如采用本书的术语,时差校正处理的输入数据$P(h,t)$被称作CMP道集(共中心点道
集),而输出数据Q则被称作CDP道集(共深度点道集)。

在实际处理中,生成该旅行时间表的第一个步骤是把深度变量变成一垂直旅行时间变
量$\tau$。所以,所要求的表是$t(h, \tau)$,为得到位于$(h, \tau)$
的输出数据,你得在位置$(h,t)$
上取输入数据。产生这种时间表的最直接而可靠的途径,看来得按$z$的步长向下推迸。实际
也就是按$\tau$的步长向下推进,并沿射线追踪。那就是说,你要针对Snell参量$p$的各种固定值,
采用对下列两项方程遍及$\tau$迸行积分的办法根据$v(\tau)$计算出必$t(p,\tau)$和$h(p,\tau)$来
\begin{equation}
\frac{dt}{d\tau}=\frac{dz}{d\tau}\frac{dt}{dz}=v\frac{1}{vcos\theta}=\frac{1}{\sqrt{1-p^2v(\tau)^2}}
\label{eq:ex3.5.4}
\end{equation}
\begin{equation}
\frac{dh}{d\tau}=\frac{dz}{d\tau}\frac{dh}{dz}=vtan\theta=\frac{pv(\tau)^2}{\sqrt{1-p^2v(\tau)^2}}
\label{eq:ex3.5.5}
\end{equation}
在式\ref{eq:ex3.5.4}与式\ref{eq:ex3.5.5}中,$dt/dz$和$dh/dz$的表达式均是以射线为基础而不是以波阵
面为基础。给定了$t(p,\tau)$和$h(p,\tau)$,为消去$p$然后求出$t(h,\tau)$,需要进行叠代及内插方
法。这听上去挺难处理的---事实确实也是如此---因为在大角度时通常存在有到达反射
同相轴中部的首波,不过,一旦完成了这项工作,你就能存该表而多次重复利用它了。各
时距曲线在大炮检距上出现交叉分支现象,推动了以波动方程为基础的速度分析方法的建
立,最大的速度灵敏度正是出现在传统的双曲线假设初单一初至假设均不成立的那些地方。


\subsection{线性性质与统计估计}
\label{sec:3.5.5}

波动方程数竭处理方法具有的线性性质使柺我们能将数据集分解成儿部分,对每部分分
别地处理,然后将它们重新组合,所得结果同未作分解时处理所得结果相同。

例如,设把共中心点道集分成两部分,比如,分成近记录道部分A和远记录道部分B。
令$(A,0)$表示各远记录道均已置零的共中心点道集,$(0,B)$可以是该道集的另一拷
贝,其中各近记录道均已置零。我们可将$(A,0)$向下延拓,然后再独立地将$(0,
B)$向下延拓。在各自完成向下延拓之后,可将$(A,0)$与$(0,B)$相加起来。我们可以轮流地
暂停延拓,就统计姓理作出某神考虑,然后选择用某种加权函数把它们组合起来。图\ref{fig:vdmo/decomp}
所示为三个记录道组成的数据集被分解成三个数据集,每道就是一个数据集,图中的半圆是
各记录道单独进行向下延拓时的扫描路径,每个半圆都通过零炮检距,给岀已适当展乎、经
过正常时差校正之后的记录道\footnote{这就是本书所谓的“波动方程时差校正”,实际就是叠前偏移。 ---译者}。

\begin{figure}[H]
\centering
\includegraphics[width=0.65\textwidth]{vdmo/decomp}
\caption[decomp]{三个记录道组成的共中心点道集按各记录道分解成三部分
左端,资料中的豚冲经过内插处理,可描绘出一双曲线轨迹。右端,各数据点均扩展成偏移半圆,
各半圆均通过位于双曲线顶点的零炮检距}
\label{fig:vdmo/decomp}
\end{figure}

采用加权函数这神思想,极端不同于我们先前进行分析的风格,它代表一种干扰识别的
概念,这是我们始终忽略了的在所有科学分析中都很重要的某种东西,那就是统计方法!是
什么因素要求选择一种加权函数呢?因素有很多,信噪比是起作用的一种因素;某些记录道
可能受噪音干扰或缺失,也是因素.当斟酌最终显示形式时,必须考虑人的感觉能力,并且
需要压缩动态范围,以便很小的值也能被发现.不但在明显易见的$(h,t)$空间内,而且在
频率域、倾角域或者在可能使波场过于超出均衡状态的任何其他空间内,都必须考虑动态范
围的压缩问题。

分解数据集有许多种方法,究竟选择何者,这决定于你的统计模型和你是否愿意使处理
重复进行许多次。也许数据道集的几部分不应该按其炮检距$h$来分解,而只应按它们的$r=h/t$
的值来分解。显然,要考虑的问题是很多的。

\subsection{共中心点道集的横向内插与外推}
\label{sec:3.5.6}

因为记录道数目不充分而产生了一些涉及到共中心点道集的实际矬理问题。截断问题便
是因检波器电缆只有一定长度并非长达遍及地震波传播之距离而发生的实际问题。图\ref{fig:vdmo/nmotrunc}
说明为什么电缆截断不但对波动方程方法,而且对常规的射线追踪和叠加方法也都是一个问
题。还有,假频问题便是因炮点与检波点彼此不是足够靠近而产生的实际问题。沿炮检距坐
标轴出现空间假频,它对波动方程方法的影响看来比对射线追踪方法的影响是个更为严重的
问题,原因在于正常时差校正降低了空间频率。因与检波器、电缆及地面耦合条件等实际问
题有关而形成的数据缺失空白,也常常会造成意外的困难。

本节将随同关于佶计缺失记录道的系统处理方法之讨论,一起解决这些问题。所述该项
技术,是斯坦福大学SEP科研项目(Stanford Exploration
Project)目前正在发展的更
为一般性的缺失数据佶计处理程序系列中最简单的一种。

首先作正常时差校正,就是说,把双曲线时距曲线拉伸展平。首先遇到的问题是:必须
用什么速度作正常时差校正才行。为易于进行记录道内插,合适的时差校正速度应该属于道
集上占优势能量的那种速度。在给定的一 个数据集上,这秤速度可以在某些时间上
是一次波速度,而在另一些时间上是多次
波速度。需用这类无物理意义的速度的原因为:必须妥善地处理强同相轴,以便保
留较弱的一些同相轴。弱同相轴的截断影晌可作为“二阶问题”而忽略不计。通
常,实际问题是要在存在有沉积地层弱反射的背景上,将海水层之强同相轴压
制掉,尤其在高频时更需如此。原则上,
我们是能够在存在强P-P波的背景上搜寻
出弱P-SV波的。

\begin{figure}[H]
\centering
\includegraphics[width=0.65\textwidth]{vdmo/nmotrunc}
\caption[nmotrunc]{以地层速度进行正常时差校正,可使良好同相轴上由电缆截
断的两点校正至良好位置,不引起作么间题•可是,绕射与多次
波的电缆截断点却校正至$a'$和$c'$,可能成为令人讨厌的问题。
这类说误所形成的干扰会造成畸变影响,使得根据共深度点叠加
求出波形时所作的精细复杂的时间序列分析全白费功夫了}
\label{fig:vdmo/nmotrunc}
\end{figure}

正常时差校正之后,残余能量应具有
很小的倾角,缺失数据之处当然除外,
这些地方用零值代替就迫使目前的数据在
空间波数域内成为宽频带了。为了改善我
们对于这种受到严重干扰的能量连续性之
视觉,我们使数据经过一稗可令高波数能量通过的"劣通"(badpass)滤波,诸如像\ref{sec:2.5}节
所讨论的高通递归倾角滤波
\begin{equation}
\frac{\frac{k^2}{-i\omega}}{a+\frac{k^2}{-i\omega}}
\label{eq:ex3.5.6}
\end{equation}
注意,这种滤波将大大地削弱具有小波数$k$的能量,即已经过适当时差校正的能量;另一方
面,在各缺失记录道附近,要注意波数纪或内的谱应是宽带的,因而这样的能量几乎是以单
位増益通过该“劣通”滤波的。

现在就从原有数据中减去“劣通”滤波的输出。这种减法要有选择地作,在存在记录数
据之处则什么也不减。这就完成了第一次叠代。然后,重复这些步骤和叠代,当在缺失记录
数据的位置上“劣通”滤波不再输出什么时,就最终达到了收敛。图\ref{fig:vdmo/missfil}所示是这种处理
的一个例子。

上述处理已忽略掉沿中心点方向存在有倾角的可能性,倾角影响要到\ref{sec:3.6}节才着手考
虑。

因为忽略了一个数据集也许会同时存在若干种速度的可能性,这种处理过程在应用方面
也是受限制的,为真正搞好这类扩展数据集的工作,可能需要有一个精确的模型以及像下一
节和\ref{sec:5.4}节中所讨论的速度谱的概念。

\begin{figure}[H]
\centering
\includegraphics[width=0.65\textwidth]{vdmo/missfil}
\caption[missfil]{左图:阿拉斯加地区有缺失空白道的野外记录;
右图:对上述记录进行递归空间滤波后的恢复结果。}
\label{fig:vdmo/missfil}
\end{figure}

\subsection{速度空间的变换}
\label{sec:3.5.7}

沿双曲线轨跡按各炮检距进行共中心点道集求和,把如此得到的一个叠加结果称作常速
叠加(constant-velocity stack
)或CV叠加。可以把速度空间就定义为是一族CV叠加、
许多速度各自都对应于一个叠加。CV叠加是一种从炮检距空间至速度空间的变换,进行CV
叠加就是根据一个$(t,h)$空间的共中心点道集来形成一个$(t,v)$
空间的速度分布图形。
常规生产性的速度依计方法相当于增添有平方与归一化处理的CV叠加。诸如CV叠加这类的
续性变换一般都是可逆的,但是变换至速度空间是属于非常高维次的变换,48道而且每道有
1000个时间采样点,这就使该种变换成为48,000维的变换了。就目前的计算机技术而言,矩
阵有这么大是没办法用代数方法求逆的,然而,却有一些出色盼近似求逆办法可供选用。

对于酉矩阵(unitary matrix)来说,转置矩阵是等于逆矩阵的。
在波动传播理论
中,一个转置算子总是满意地近似于一个逆算子。Thorson(l984)曾经指出,对CV叠加的
转置运算就跟CV叠加本身没什么两样。要对CV叠加作转置运算,得从一个速度分布图形开
始作起,即从一个$(t,v)$空间中的图形作起。为形成某个给定炮检距A时的原始状态,首
先必须强使$(t,v)$图形中的每一个记录道恢复原来的正常时差;亦即,必须迫使同相轴从
$(t,v)$图形中所具有的零炮检距时间$t_0$恢复到与给定炮检距$h$相应的时间上去。然后遍及
所有速度$v$将该种$(t,v)$
图形进行叠加,以产生该给定炮检距形下的地震记录。对所有
的待求炮检距的值A重复迸行上述恢复过程,即可恢复原来的共中心点道集。除了把时差
校正拉伸公式改变成对压缩进行补偿之外,对CV叠加进行转置运算的程序同CV叠加本身的
运算程序很相像。

% 浅水干扰可能由沉没船只的散射波所形成,或者是距测线若干公里之遥的某个岛屿或冰
% 山的侧面所散射的波。想一想散布在整个浅海海底的漂砾吧,它们不但沿着船的航道分布,
% 而且沿其两側散布。由这些漂砾形成的反射时距曲线正好适甩随机点散射模型。由于地震波
% 具有较长的波长,接收设备使我们无法将这些侧反射与上、下行波加以区别。

% 试想像位于船的一侧有若干公里之遥的这些浅层散射体之一,更准确地说,应令该散射
% 体处于海底表面上,位于垂直通过炮点与检波点连线的中点之直线上。对于这一个散射体来
% 说,其共中心点道集应是一精确的双曲线,犹如是图\ref{fig:ofs/randcsp}上之深层反射体所形成的一般,
% 既然这是一个由水层速度决定其形态的双曲线,以沉积地层较高的速度进行的共深度点叠加
% 方法应当能很好地压制掉这类散射干扰。所以,以前述及的产生“坑蓆状干扰”\footnote{Larner Streaks}的散射不是侧
% 向散射,“坑蓆状干扰”的散射是由那些沿着测线分布的散射体所形成,而不是由垂直于测
% 线方向分布的散射体所形成。

% \subsection{偏移椭圆}
% \label{sec:3.2.8}

% 另一种深刻了解方程\ref{eq:ex3.2.3}的办法是把炮检距$h$和总旅行时间$t$看作固定常数,这时
% 最终可证明,可能的反射点的轨迹在$(y-y_0,z)$平面上应是一个椭圆。之所以为椭圆的
% 原因可由椭圆的几何意义得出。要画出一个椭圆,可将一个钉子或图钉固定在图\ref{fig:ofs/pgeometry}的
% $s$处,另一个则钉于$g$处,用一根线将图钉联结起来,线的长度应足以从$s$经$(y_0,z)$至$g$。
% 用一支铅笔沿该线滑动同时保持该线是绷紧着的,那就可以作出一个经过
% $(y_0,z)$点的椭圆了,该线应使总距离化保持为常数。

% 零炮检距时间剖面偏移的一种方法是绕射扫描,取$(y,t)$空间内每一个数据的值,然
% 后用它作出$(y,z)$空间内的一个适当的半圆。在非零炮检距情形下,应将该圆推广为椭圆。

% 要证明可将方程\ref{eq:ex3.2.3}变成椭圆、即拉伸压扁的圆的标准数学形式,可不大容易,
% 但是这个结果对于以后的分析有简单而又重要的意义,所以我们必须在这里证明一下。在
% $(y,h)$空间内,式\ref{eq:ex3.2.3}为
% \begin{equation}
% tv=\sqrt{z^2+(y-y_0-h)^2}+\sqrt{z^2+(y-y_0+h)^2}
% \label{eq:ex3.2.6}
% \end{equation}
% 为有助于减少代数上的累赞,现定义新的$y$,使之等于原有$y$再减去$y_0$,即$y\rightarrow y\rightarrow y_0$再作
% 下列定义
% \begin{subequations}
% \begin{equation}
% tv_{\text{rock}}=2d=2tv_{\text{half}}
% \label{eq:ex3.2.7a}
% \end{equation}
% \begin{equation}
% a=z^2+(y+h)^2
% \label{eq:ex3.2.7b}
% \end{equation}
% \begin{equation}
% b=z^2+(y-h)^2
% \label{eq:ex3.2.7c}
% \end{equation}
% \begin{equation}
% a-b=4yh
% \label{eq:ex3.2.7d}
% \end{equation}
% \label{eq:ex3.2.7}
% \end{subequations}
% 利用这些定义,将式\ref{eq:ex3.2.6}变为
% \begin{equation}
% 2d=\sqrt{a}+\sqrt{b}
% \label{eq:ex3.2.8}
% \end{equation}
% 两端取平方后,得出只具有一种平方根的新方程
% \begin{equation}
% 4d^2-(a+b)=2\sqrt{a}\sqrt{b}
% \label{eq:ex3.2.9}
% \end{equation}
% 再取平方以消去平方根
% \begin{subequations}
% \begin{equation}
% 16d^4-8d^2(a+b)+(a+b)^2=4ab
% \label{eq:ex3.2.10a}
% \end{equation}
% \begin{equation}
% 16d^4-8d^2(a+b)+(a-b)^2=0
% \label{eq:ex3.2.10b}
% \end{equation}
% \label{eq:ex3.2.10}
% \end{subequations}
% 引入$a$与$b$的定义,则
% \begin{equation}
% 16d^4-8d^2[2z^2+2y^2+2h^2]+16y^2h^2=0
% \label{eq:ex3.2.11}
% \end{equation}
% 置$y$与$z$于右端
% \begin{subequations}
% \begin{equation}
% d^4-d^2h^2=d^2(z^2+y^2)-y^2h^2
% \label{eq:ex3.2.12a}
% \end{equation}
% \begin{equation}
% d^2(d^2-h^2)=d^2z^2+(d^2-h^2)y^2
% \label{eq:ex3.2.12b}
% \end{equation}
% \begin{equation}
% d^2=\frac{z^2}{1-\frac{h^2}{d^2}}+y^2
% \label{eq:ex3.2.12c}
% \end{equation}
% \label{eq:ex3.2.12}
% \end{subequations}
% 最后,代入以前定义所有的定义,则
% \begin{equation}
% t^2v_{\text{half}}^2=\frac{z^2}{1-\frac{h^2}{t^2v_{\text{half}}^2}}+(y-y_0)^2
% \label{eq:ex3.2.13}
% \end{equation}
% $t$值固定时,式\ref{eq:ex3.2.13}就是$z$轴已伸长了的一个圆方程。我们进行的代数运算证实:藉助
% “图钉与线”所定义的椭圆同“拉伸压扁的圆”之定义是一致的。
