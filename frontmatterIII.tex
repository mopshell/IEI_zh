{\ctexset{section/format+=\centering}\section*{序言}}

反射地震学家是要获得地层内部的映像。直到60年代为止,成像都是按一种特定方式
实现的。在1968年至1972年这段时期内,我构思出一种直接基于波动方程的新型成像方
法,并经过了野外观测的检验.以前都是从简化了的假想模型出发,利用波动方程来预测观
测结果,而在常规的数据分析中则未采用波动方程。我提出的有限差分的成像方法在石油勘
探工业中很快就得到了推广应用,随后,有许多其他人士也迅速参与其事,并作出了一些重
要改进。早期的特定成像方法经过重新解释,亦按照波动理论加以完善了。

由工业部门资助,在斯坦福大学成立了一个名为\textbf{SEP}
(斯坦福勘探计划)的小组,专门从事这项技术的研究。在48个资助单位中,许多单位都设有自己的具有良好素质的研究部
门。70年代中,取得许多进展的十年就这样开始了,在80年代,继续大步前进。

写作本书之目的萌生于向进入勘探领域的新手讲授这许多新概念中最精采部分的需要,
考虑到来自地球物理领域以外的人很多,我已使地球物理专门术语保持至最低限度,并且对
每个术语均作了定义。因此,料想本书不仅对有志于石油勘探的人们,而且对所有从事波场
分析之学科领域中的专业人士都将有所裨益

我以前写的《地球物理数据处理基础》\footnote{该书的中文译本于1979年由石油化学工业出版社出版.——译注}一书出版于1976年,最近业已重版。该书内容
涉及反射地震数处理的很多基本问题,诸如\textbf{Z}变换、傅氏变换、离散线性系统理论、矩
阵、统计学,以及厚状地层构造理论等等。该书也介绍过波动方程成像方法,然而,现在看
来,进行广泛增补已变得很有必要了,本书就是由这些增补演化而成。这两本书之间的不同
之处约占90咏,重复的部分占10\% 。这是出于使本书能独立成书之目的。

本书对石油勘探中所采用的数据处理技术的整个领域进行了紧凑精练的总结回顾,它是
斯坦福大学勘探地球物理课程中的基本教科书,不过,我并未奢求这本书是包罗万象的百科
全书,对于反褶积与静校正这样一些重要的处理方法,只是略加叙述,一带而过,对于层析
成象方法的实际应用这类内容亦复如是,而其他一些技术方法,诸如射线追踪(参阅\textbf{Cerveny},1977)及许多种类的正演模拟方法等,则均略而未提。很遗憾,仅有关偏移方法的文
献就已够浩瀚的了,以致对象\textbf{Kirchhoff}偏移方法在理论方面的一些显著贡献(参阅\textbf{Berkhout}, 1980 )也不得不舍弃了。

地震成像是一项从数学与物理学中汲取了许多内容的课题,这些题材按照一种逻辑顺序
将一项概念建立在另一项概念的基础之上;我是按类似顺序选择组织本书内容的,这样组
织内容有利于想透彻理解材料内容的新入学的学生。四处散布的一些有关实际问题难免为逻
辑结构所遗漏,为此,书末附有详细索引和上百篇涉及多种问题的参考文献,以供读者查
考。

更为广泛的论述反射地震学的专题教科书有\textbf{Waters} (1981
)与\textbf{Sengbush} ( 1983)的著
作。着重于石油勘探方面的描述性讨论反射地震学的书,有\textbf{Sheriff} (
1980 )与\textbf{Anstey} (1980 )等的著作。
\textbf{Aki}与\textbf{Richards} ( 1980)以及\textbf{Kermett} (1983)诸人写的书是关于天然地震学方面的补充读物.

我还试图使本书适合于那些想学习这些概念而对数学只想作略读的读者们的需要,个别
章节(各节都是一讲)在进行数学分析之前都尽可能地使之包含有实际问题的描述说明,各
章本身也是按此方式安排,因此,比方说,在你读第一章读到中途时,你可以跳过去直接去
读第二章。

波动现象恰好是美妙的几何学研究对象,利用很少一点数学分析就能学到很多内容,但
是,你应该是事前已经通晓了微积分、复指数和傅里叶变换等基本知识,之后再开始读本书
为好。

理论与实践之间总是存在有“空白”的。许多书都没给你提供有关“空白”之所在及其
范围的确切线索------即使是勘探地球物理方面的书籍也不例外。对这种“空白”,没必要感
到困惑不安,这正是研究课题活力所在------任何科学具有发展活力都是如此。“空白”就是
一个活动目标靶子,它涉及范围的大小就看你采取什么观点研究它了,所以,我得冒些风险
告诉你,该作什么,不该作什么;什么是重要的和什么是不重要的。观点总是超越于事实
的。你要是既不了解某些观点,又不了解实际,你的知识就不会是完整的知识。当我解释说
明理论与勘探实践之间的脱节时,以及当解释说明应该作什么而看来还没这样作的时候,你
将会既获得观点又获得事实\footnote{在中译本中做了节译。——译注}。




