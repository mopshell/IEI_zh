石油勘探是从地震测深开始的。用计算机将回声处理成可以揭示出许多地质历史的映
像。就全球范围而言,回声测深与映像处理就构成了大约每年40亿美元的经济活动。


\textbf{1.观测的意义}


石油与天然气的存在,对地震反射的直接影晌很小。岩石体积比烃类的体积要大很多
倍,不过,利用各种不同类型岩石之间的分界面,是可以很好对比追踪反射的。多孔隙岩石
中的烃类可以自由流动,流体有升起的趋向,岩石分界面的形状可以告诉我们什么地方可能
有烃类聚集。在北海中部发现石油与天然气是反射地震方法极为成功的一个例子。当以反射
地震学方法确定井位的第一口勘探井正在钻进之际,谁也不可能预料他们是否会钻遇石油,
可是,一旦当北海下面不论什么地方发现了石油,人们就会对如此确定的井位比随机定位的
那些井确实是更有利得多这一点抱有更大的信心。事实证明,的确如此。

在一口井已经钻完并完成了测井之后,反射映像就变得更为有价值了,因为根据它就可
以知道相应于每个回声反射应是什么岩石类型。地震学通常能够提供关于距井若干距离之处
岩石类型的出色的精确图像,特别有价值的是可以了解岩石沿什么方向上倾和地层在什么地
方被断层所断裂。地震学提供这种信息,其费用比多数钻井费用低得多。在转移至海上进行
石油勘探时,地震费用降低一个数量级,而钻井费用则要升高一个数量级。


\textbf{2.观测资料的可重复性}

反射地震资料数量庞大,它不是用铅笔在一张纸上画的记号,而是一盘接一盘的高密度
磁带。有些地震资料很容易理解,但是有许多却不那么容易,尤其是初次试验时的资料。虽
然许多资料不易理解而且看上去还有噪音和随机干扰,但值得注意的是这种资料在试验中是
可重复的。我们发现,用这种资料进行工作,可以认识到越来越多的东西,于是受到鼓舞而
继续做下去。因为在以常规方法采集的数据中仍然有许多信息隐藏其中,所以本书主要集中
于占主导地位的野外数据观布置,即通常具有近地表震源与接收器的单次测线。各种观测
技术仅作说明而不加以研究。


\textbf{3.作为成像工具的计算机}


哲学家提出问题:``何谓认识?''
。作为技术人员,我们的回答是:只要现实世界存
在,在我们心中也就存在它的一个映像。所谓认识,也就是意味着这二者是相似的。为有助
于形成映像,我们使用了显微镜、望远镜、辑影机、电视机等等这样一些成像装置。在本书
的描述中,计算机则是地震回声测深的成像工具。

计算机作为成像工具,在许多方面是颇为理想的。望远镜是受其组成部件质量限制的,
而计算机所形成的映像,很大程度上是受我们对数学、物理学和统计学的理解深度所限制,
而不是受计算机内在特性的限制。要是用雷达或者超声波来成像,计算机容量就会成为一个
现实问题,可是地震回声测深的信息含量(频带宽度)正好是大致与当今计算机的容量相匹
配。


\textbf{4.为何有趣?}


许多年轻人似乎都以揪住觫手的理论问题不放为乐事,可是一旦到了需要实际应用的时
候,他们往往失望地发现,这个理论在某些方面是离题的,或是不适合于当前的问题,一开
始的时候,这会减少对于实际问题的兴趣,但是最终许多人会达到这种境界:把实际问题看
成是比原有数学模型更为有兴趣的事。为什么会这样呢?

生活也许是像一种计算机游戏,我曾经注意到,学生们最喜爱的游戏并不是那些具有一
种预先决定的内在逻辑结构的游戏,他们喜欢可以允许他们在玩游戏时能逐步揭露其规律的
游戏。当由于应用自己个人的一些概念而使游戏老受挫折的时期能够告终,那确实是乐趣无
穷的。可是,要成为有趣的事,游戏就必须有若干规律,而你经过相当数量的努力必须能够
揭露出它们。很幸运,反射地震学连同现代计算机,就为我们提供了这样一个类似的环境。
有时,游戏可能总是失败,这时需要别人给你一点提示,使你克服某些障碍,进入一个新的
境界,达到较高的水平。阅读这本书并不像是玩这种游戏,它倒更像是给你题解大全或是锦
囊妙计,有助于你达到更高水平。

这些策略妙计大多数是新概念,其中许多概念的形成时间都不超出十年,所以选择它
们,是因为它们确实起作用,虽然并不总是、但往往是可令人满意的。我已经抑制了想把许
多虽尚未经充分试验但是颇有发展前途的策略妙计包括在内的强烈愿望了。

实践问题不但是比理论问题更深入一层,而且从根本上来说它们还会产生更为有意义的
理论。例如,我在大学一年级物理实验室内曾经想根据简单试验导出牛顿定律,我应从实验
中发现力等于质量乘加速度,当然,我没发现事实确实就是如此,试验似乎进行得并不顺
利,因为还存在有尚未考虑到的摩擦力。对你来说,现在摩擦确实就是一个有趣的主题了,
物理学家、化学家、冶金学家、地球科学家,全都了解牛顿定律,但愿他们都懂得摩擦
力!

你们现在所拥有的理论书籍,除两种较早期的理论处理方法如:成层介质数学物理理
论、时间序列分析之外,都还没有写出,也不可能涉及到我们数据资料中的一些最有趣的方
面。有的人认为我们仅仅有涂改玷污的数据!反射地震数据资料是可以重复回放的,我们的问
题中有许多实际是从理论产生的,而不是从数据产生的。


\textbf{5.计算机与电影}


这本书包括有一系列计算机程序,这些程序是用于解说例子和作为练习,也可作参考之
用。尽管还不能保证它们是尽善尽美,但是我作这本书中的许多图件时,它们曾起过作用,
因而它们应当也能为你服务。你将注意到,这是类似于FORTRAN的一种程序语言,在1.7
节开头部分就有这种程序的叙述说明。由于每人备有不同类型的图形输出装置,你自己要想
采用这种程序,那你就得精通这类装置,以便能将它们的输出接通至你的绘图设备。

电影实际就是许多画面的集合,在一台计算机中,它只不过就是一种必须以某种方式使
之转换为光亮画面(图形元素)的浮点数三维矩阵。目前,少数人已配备有可将这样一种三
维矩阵直接转换成电影的设备。在我的实验室中,这种转换是在一种高质量录像计算机终端
(AED512)上完成的。电影的潜在能力是一种很有价值的财富,它增强了我们对数据资料
的理解和对数据处理的理解。学生们因亲眼见到自己进行的程序工作能立即形成可用录像磁
带转录的一部电影,从而大大受到鼓舞。与其它图形显示装置相比,这种装置很容易维修
保养,进行研究工作的学生和攻读硕士学位课程的研究生都会使用它,利用它作课处作业练
习。

包括快速直接存取(DMA)计算机接口在内,这类设备的费用不超过一万美元。就真
正有效地采用这类显示方法的经验而言,你还应当具有可对内存大于几兆字节的一种计算机
进行物理控制的手段,如果你不是已经具备这点,处理成本费用将会增加大约十倍。


\textbf{6.有事可干吗?}


反射地震成像方法主要应用于石油勘探,烃类不同于核能,它是一种非再生能源,而且
还有迹象表明,在年轻一代有生之年期间,石油生产一定会下降。这是不是就意味着年轻人
应回避研究这方面的问题呢?我想,当然不是如此。就长期的观点来看,随着地球上人口的
继续增加,很难想像人们会失去对地壳进行研究的兴趣;就较为中期的观点看,随着能源蕴
藏丰富程度的减少,势必激起更大的勘探能涵的努力;就短期的观点看,从事能源勘探的工
作者在今天是很需要的,而且现在还不存在以煤或核能为能源的飞机。在任何情况下,本书
所讨论的技巧、物理概念在计算机上的实现方法等,将始终是具有普遍适用意义的。


\textbf{7.本书阅读指南}


第一章与第三章阐述反射地震学成像的基本概念。第二章与第四章的内容为分析被观测
波所需之计算机方法技术。第五章阐述先进的成像概念。在斯坦福大学,第一章至第三章悬
硕士研究生一个学期的讲授课程内容,在选修根据本书开课之前或之后,这些学
生同时还选修一门根据《地球物理数理基础》开出的课程。你也许想不学习有关方法技术
就能对概念有所理解,那你就不妨试着只读第一章和第三章容,不过,第二章内容因其具
体性质和它所包含的例子,将会增进你的理解能力,不妨也读一读。第四章内容适合于那些
想了解高质量完成任务时涉及到一些什么问题的技术人员,或者适合于那些希望通晓各种方
法的技巧与精度限制的非常熟练的解释人员。第五章阐述一些新颖的成像概念,它们在原理
上看来是正确的,但是因各种并非全都能为我所知的原因,它们均尚未获得广泛实际应用。
容忍得了数学的解释人员也许会赞赏第五章,因为这一章的宗旨就是解释事情是如何积为什
么总是按他们在实践中作的那种方式而完成的,不过,对于那些希望去发展新型回声成像方
法技术的人,第五章内容才真正会具有主要的吸引力。

