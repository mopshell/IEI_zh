\section{二维傅氏变换技巧}
\label{sec:1.6}

本节内容是对那些将要从事偏移方法处理的人们有用的提示大全。

\subsection{傅氏变换中的符号与比例因子}
\label{sec:1.6.1}

在进行$t$坐标轴、$x$坐标轴与$z$坐标轴的傅氏变换时,必须对每一种坐标轴选择一项符
号约定。电气工程师选择了一种约定,而物理学家选择了另一种约定。虽然二者的选择均有
良好的理由,可是我们所处的环境更为类似于物理学家的处境,所以将采用他们的约定。对
于逆傅氏变换,这就是
\begin{equation}
p(t,x,z)=\iiint e^{-i\omega t+ik_xx+ik_zz}P(\omega,k_x,k_z)d\omega dk_x dk_z
\label{eq:ex1.6.1}
\end{equation}
对于正向傅氏变换,空间变量应带有负号而时间变量则带有正号。连续情形下的积分限与比
例因子不同于离散函数情形。从解析上说,我们很少在单纯任何一种情形下完成变换,由于
积分限与比例因子所需额外的符号通常会増加混乱而不是使讨论更清楚,所以除在它们可起
有用作用时以外,式中的积分限与比例因子将全部略去。

符号约定非常重要。因为有很多空间坐标轴(以后还要引入中心点坐标和炮检距空间坐
标并进行相应的变换),所以建立一种符号约定是必要的,有些人对符号采用试验选择法,这
多半会因可能的排列组合数目太大而使问题复杂化。我们有充分的理由采用物理学家所选择
的符号约定,而且一旦了解了这些理由,很容易记住这些约定。

根据约定,波应沿空间坐标的正方向运动,将空间坐标取为半径时,这点尤其明显。像
地球物理震源那样,原子总是从一个点向无限远而不是沿其他路程辐射能量,所以我们将约
定总是选择在任何空间轴上沿正向传播的波。在式\ref{eq:ex1.6.1}中,这点意味着空间频率的符
号必须与时间频率的符号相反。这个说明既适用于正变换也适用于逆变换。

现在还剩下一个究竟是对时间坐标取正号还是对空间坐标取正号的问题。空间坐标有许
多个,可时间坐标却只一个,如果选取空间梯度$\partial/\partial x$、$\partial/\partial z$如等使之相应于正的$k$向量,即相
应于$ik_x$、$ik_z$等,那么,负号数目就最少而且符号改变最少。当然,这就只剩下使时间导数
相应于$-i\omega$了。

这种符号约定使我们的处理习惯正好与电气工程师的处理习惯相反,他们很少处理与空
间坐标有关的问题,很自然就选择了使$\partial/\partial t$与$+i\omega$相应。据我所知,采纳电气工程师的选择,能
列举出的最佳理由只不过是因为我们是利用电气工程师采用微程序编码所设计制造的阵列处
理机进行计算,这时工程师们当然是使用他们自己的符号约定。不过这对将复值时间函数变换
至复值频率函数的程序编制无关紧要,因为这时符号约定是在用户控制之下,但是对于将实值
时间函数转换为复频率函数的程序,这会造成一些差别。既适用于实值域又适用于复值域的
两全其美的办法是:把程序所产生的频率范围想像成不是如程序说明的样从0至$+\pi$而而是
从0至$-\pi$。再一种办法是,你总可取变换的复共扼,它将改变$\omega$轴的符号。采用Stolt偏移
算法时,普通都是首先完成空间变换,结果,阵列处理机的约定最终就同我们的记号一致
了。

\subsection{大矩阵如何转置}
\label{sec:1.6.2}

非常大的矩阵幸好可以很容易转置,正是这点才使得在小型微机上进行波动方程地震数
据处理是可行的。所谓非常大的矩阵,我的意思是指大到计算机随机存取器容纳不了的一种
矩阵。如果随机存取器容得下两倍的数据量,那么转置不过就是取拷贝运算$T(i,j)=M(j,i)$。

对于非常大的矩阵,转置算法既简单又策略,因此,我将用一种纸牌策略为例来描述
它。我手上有一副牌,从中去掉九点、十点及$K$、$Q$、$J$等人头牌。令$a$、$b$、$c$与$d$分别代表红
桃、黑桃、梅花与方块,然后我钯这些牌按下列次序排列($A$牌用1表示)\\
la、1b、lc、1d、2a、2b、2c、2d、3a...... 、8d\\
现在我发牌,顺序交替使牌面朝上,一种垒成$A$堆,一类垒成$B$堆,你瞧:\\
$A$堆:la、lc、2a、2c、3a、3c、......8a、8c;\\
$B$堆:1b、ld、2b、2d、3b、3d、......8b、8d。\\
其次我把$A$堆放在$B$堆上面($A$在$B$之前),然后再顺序交替发牌,分成$A'$堆和$B'$堆,你瞧:\\
A'堆:la、2a、3a、...... 、8a、1b、2b、...... 8b;\\
B'堆:lc、2c、3c、...... 、8c、1d、2d、...... 8d;\\
现在我把$A'$堆放在$B'$堆上面。当我们开始玩牌的时候是所有的一点在一起、所有的二点在
一起等等,而现在已成为所有的红桃在一起、所有的黑桃在一起$\ldots\ldots$等等。因此,你瞧,只
不过发两次牌,我就把这副牌转置了。原则上,转置矩阵的这种算法只需四盘磁带,几乎无
需磁心存储器。

现在来试一下相反方向的转置。注意,这次我得三次发牌而不是两次发牌,才能恢复原
状,这是因为这副牌的红桃、梅花等共有$2^2=4$类,从一点至八点共为$2^3=8$种点。实际上,
还有另一种算法能允许我只需停止叫牌两次而不是三次,就能完成相反的转置。按照这种算
法,你只需把每件事倒过来作就是了。先是从$A'$堆和$B'$堆开始,轮流交替地从$A'$堆中取一
张牌,从$B'$堆中取一张牌,于是就可形成$A$堆;按类似办法再形成$B$堆,然后,重复这种过
程,直至恢复原状。在这种处理过程中,第一种算法称作分类算法,而第二种则称作排序算
法。用这两种算法,对一个大小为$2^n\times 2^m$的矩阵可以经过$m$或$n$次(按其中之较小者)处理,
即可完成矩阵转置。

还有许多可能的推广方法。把牌分成四堆,就可以建立处理维数为$4^n$的矩阵转置方法,
这将减少停止叫牌的次数,但却要求増加磁带驱动器台数。类似地,还可以将任意顺序分解
成若干质数顺序,等等。但是,这样讨论就离题太远了。

使停止叫牌的次数极小,其结果就是使磁带数目极大。实际处理中,当你进行矩阵转置
时,你并不会利用真正的磁带,其实,你是在一个大容量磁盘上模拟那礙带操作运算,因
此,你所选择利用的“磁带”数目将受随机传输速度对顺序传输速度之比值所控制。

\subsection{无需进行转置的罗卡(Rocca)二维傅氏变换}
\label{sec:1.6.3}

在计算机中完成二维傅氏变换的最直接方法就是重复应用一维傅氏变换。最容易的部分
往往是“最快速”的方向,就是说,如果数据矩阵是按列存储——利用FORTRAN程序语
言时即如此——则列变换就是重复使用一维变换程序的常见操作。现在讨论行变换。如果把矩
阵输入于随机存取器,则每件事就容易作了,可将某个时间上的一行元素作为一个向量处
理,对该向量进行傅氏变换,然后置于矩阵的该行。最典型的情彤不是把数据输入于随机存
取器内,而是输入于“虚内存”,这意味着程序人员能写入$T(i,j)=M(j,i)$,但程序运行却
将极慢,因为从磁盘取出整整一页的虚内存才只求出一个数。

从概念上说,沿行的方向处理傅氏变换的
一种比较容易的办法是将矩阵转置、对每列进
行变换,然后再转置回去。富比奥•罗卡(Fabio Rocca
)曾提出一种快速而又容易的按行
的下标完成傅氏变换的方法。基本的傅氏变换程序都有一定数量的常规计算,诸如计算或调
用正弦和余弦。一般来说,执行一个傅氏变换就得每次重复这些常规计算。采用罗卡的方
法,则只要完成一次这些常规运算,就可使所有的行均完成傅氏变换,因此,它甚至比直接
方法还快。罗卡的方法如下所述。

可将数据矩阵看成是一种行向量,其元素由每一列所组成,在各该列内按下标从小到大的
顺序取数,可以在行运算之前或之后用一维傅氏变换将各列加以变换。要完成行运算,只需把
普通的傅氏变换程序修正一下即可。办法就是把对行进行的每种标量加法与乘法运算改成对
相应列内每一个元素进行相同的各种运算。

数据的存取顺序使罗卡的按行算法在虚内存条件下有很高效率。在具备了现今的虛内存
以前,我们是采用环绕着内循环进行读出与写入的办法来实现罗卡的按行算法的。为说明罗
卡方法,曾经根据《地震数据处理基础》一书中的一维傅氏变换程序编制过一种按行进行傅
氏变换的程序,该程序可将复值时间函数变换为复值频率函数。如果你诀定要编制一个从实
值至复值的傅氏变换程序,你对实部与虚部要邻接存储的设想应当提高警惕,对于列下标,
这种设想是成立的,但是对于行下标却并不成立。
