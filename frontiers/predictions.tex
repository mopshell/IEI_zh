\section{未来十年的展望}
\label{sec:5.8}

地震学家在20世纪60年代已经熟悉如何将时间序列最优化理论应用于地震数据了。该种
理论可参阅《地球物理数据处理基础》一书。因其处理空间关系的办法过于简单,时间序列
方法终于达到了走下坡路的转折点。在70年代,地震学家学会了应用波动方程,这就是本书已
经涉及的内容。你能够看出,应用波动方程这件工作尚未结束,可是我们总算已经走过了很
长的一段路,也许我们已解决了大多数“一阶”问题,因而剩下的主要是“二阶”问题了,
为了使二阶效应能有意义,必须恰当地考虑到所有的一阶现象。

本书论及的一些一阶效应仅是稍微涉及到地震数据中既明显又微妙的不完善性。

\subsection{数据库问题}
\label{sec:5.8.1}

我们经常遇到截断(truncation)问题。记录电缆当然是长度有限的,因而可觉察的波
一般都是传播距离完全超出它的,地震勘測本身就具有有限维数。我们也会遇到数据空缺
(gaps)问题,地震数据中可能出现空缺是不可预测的,像汽枪起爆失误或者勘探人员在勘
测中途无法接近陆地矿区取数等就是如此。此外,我们还有空间假频问题。由于技术的进
步,我们可望能显著减少检波点坐标轴方面的假頻,但是炮点坐标轴上的假频将仍存在。一
天只有廿四小时,而每次激发我们必须间隔十秒钟,以便回声反射能消失,否则就会重叠。
因此,给定一定勘探面积和一定数量的勘探工作月数时,我们在每平方公里内窬有一定数量
炮点才能完成勘探。就海上资料而言,同数据远离测线分布时出现的那些数据问题相比,沿
勘探船航跡路程之测线内的分布间距是不存在什么问题的。

偏移可提供苁数据空间至模型空间之映象,这种变换是可逆的(在非耗散子空间内)。
当缺失数据时,变换矩阵被分成两部分,一部分作用于已知数据值,另一部分作诏于缺失数
据值,除了\ref{sec:3.5}节提到一点以外,本书均忽略缺失部分。虽然\ref{sec:3.5}节内提出了处理缺失部分的
策略,但办法非常费时间,因而我相信它最终将被取代或者被大大改进。

有干扰的数据可定义为不拟合于我们模型的数据,例如,如果用零值代替缺失数据,该
资料就不能看作是完整资料而是有干扰的资料。在信噪比已知为零之处,就是数据缺失之
处。这问题还同更一般性的噪音模型有关,但是局部相干多维波场的统计处理方法无论在理
论上还是实践中均尚未得到很好的发展。

我的预测是,未来十年的主要研究活动将会是力图学会同时处理波场的物理性质和统计
性质。

\subsection{将最优化理论同波动理论重新结合}
\label{sec:5.8.2}

让我们超出本书窥视一下未来.地震影像典型情形是一种$1000\times
1000$的平面,这个平面
是由一个大约为$1000^8$个相互联系之数据点所组成的体积导出的;处处存在有未知因素,不
但在地层模型中有,而且在数据中也有,如噪音、如数据缺失,以及如空间采样密度不充分
和数据记录范围有限。为综合解释,我们必须把来自物理学的原理和来自统计学的原理相互
结合起来才行。假设以某种神奇的最优化公式能够作到这点,查一查最优化理论就能证明,
在经过次数大于未知数个数的若干次叠代计算之后,求解方法肯定是收敛的。从而,一旦我
们了解如何洽当地提出问题,问题的求解看来就得要求现行计算机的计算能力提高一百万倍
才行,这可真是个了不得的问题!

但是,你越考虑这个问题,你越会对它感到兴趣。首先,我们有个最优化问题要解决;
其次,既然我们受限制只能进行少数次叠代计算,比如说,只进行三次叠代,那我们就必须
在这仅有的三次叠代过程中尽可能地得到更多信息。现在,我们不但有原来的最优化问题,
而且还有一个以最优方式求解这个新问题需要解决。为此,我们首先可以利用原始数据内有
关联的随机性,然后,在最优化计算过程中,逐次叠代时均以有关联的随机方式修正改变地
层模型,直至得出最优解。这第二种最优化问题不仅仅是一个实际问题,在理论水平方面它
也是个比较深刻的问题。

\subsection{扔掉你的纸面上的剖面}
\label{sec:5.8.3}

目前的地震解释工作往往就是手拿彩色铅笔在计算机产生的图象上点点画画,使其面貌
増强清晰程度。地震解释正在进入一个新时期,将要在磁带录像机屏幕上完成解释是这个新
时期的特点,所以要如此作的基本原因就是因为一页纸只有二维,而大多数反射数据却是三
维的;现代三维地震勘探实际上是四维数据的记彔,彔像机屏幕可以把它作为一种电影来表
演,操作人员或解释人员可以利用这秭电影来实现人机对话。有许多东西我想表演给你看,
但是我在一本书内不可能作到这点。地震资料或者甚至一张空白的纸都是有其结构特征的,
当移动一个有其结构特征的物体时,你立刻能识别出它来,但是我用本书中的静止图形就不
可能使你得到相同印像了。眼睛在两张图形之间作任何运动,是会妨碍眼睛对微小变化的感
觉能力的。天文观测人员采闱观察不同时间所摄相片的办法来搜索天空的变化,他们每观察
完一张就迅速眨一下眼睛,然后再观察下一张,采取这种办法就可得到活动电影的感觉。我
们的眼睛是特种专用计算机。活动电影常常是表现“某物来自何处”这类内容,能够使我们
注意到一般环境背景中意料不到的东西。

大多数地震解释要在叠加剖面上完成,原始数据是三维的,但由于求和过程使其减少了
一维。从理论上说,求和过程只是消除了冗余信息而却増强了信噪比,但实际上事情远比这
复杂得多,在用人类肉眼完成求和时(只是以加速电影画面移动速度的办法来完成求和),
能感觉到东西将会多得多。因此,将会存在两代地震解释人员,一类是注视着他们的录相机
屏幕、能够解释叠前资料的那些解释人员,而另一类则只能解释叠后剖面。