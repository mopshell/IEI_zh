\section{波场外推的物理问题与修饰处理问题}
\label{sec:4.1}

频率滤波、倾角滤波以及增益控制看来是目的在于进行大量修饰的三种处理:它们全都
是用于改善地震记录面貌。选择这些处理或类似处理的定量参量时所采用之准则往往是含糊
不清的,而且与人的经验或视觉有关。原则上,乞灵于信息论并采用像信号与噪音倾角普之
类的客观准则,应该有可能选出所需的参量,但是在常规的实际工作中,迄今还没这样作
过。

不应当低佶修饰性处理的重要性,例如,想对一些处理技术方法进行比较,在许多场合
下往往就因为修饰性参量发生意外的变化,而导致失败。要求有这些修饰性处理是在波动传
播理论内自然而然产生的,所以,看来最好是首先理解它们是如何形成的,然后在处理过程
中进行这些修饰,而不要在处理以后才试图以某种人为方式生硬地附加上去。现在将对波场
外推方程各个部分逐一加以研究,以指出它们的修饰性效果。



\subsection{$t$平方函数}
\label{sec:4.1.1}

反射随时间而变弱。为能看清较晚时间到达的数据,我们一般要使数据的放大倍数随时
间而增大。我始终很少因我选取$t^2$函数作为比例因子而感到失望过。不可能总是期望产比例
函数行得通,因为它到底是以非常简单的模型为基础的。但是我发现$t^2$比流行的抉择、即选
取增长指数函数,更令人满意。$t^2$函数没有什么参量,而指数函数则要求有两个参量,一个
是时间常数,另一个是截止时间,也就是到达这个时间你必须停止指数增长,因为它会变得
过于大了。

$t$取二次幂有两个原因。第一个原因是因为我们正在把三维问题变换为一维问题。地震
波是在三维空间内扩展的,不断扩展着的球面波阵面表面面积与半径之平方呈正比关系,因
而能量分布面积隨时间之平方而正比增大:但是地震波振幅是与能量之平方根成比例的,所
以由能量分布的这种基本几何形态关系可预言球面发散校正仅需时间的一次幂。

第二个原因是由计算简单的吸收而引起。讨论吸收作用要有某种模型,我将提出的模型
对于解释有关地震波吸收的任何事情都显得过于简单了,可是它却能很好地预言应有另一个
时间的一次幂,而经验证明我们确实是需要如此。关于该模型,我们假设:
\begin{enumerate}

\item
  一维空间传播;
\item
  速度恒定;
\item
  吸收Q\textsuperscript{-1}为恒定;
\item
  反射系数沿深度方向为随机的;
\item
  不存在多次反射;
\item
  震源为白噪音。
\end{enumerate}

这些假设直接告诉我们:单频波将随深度而指数式衰减,比如说,按$exp(-\alpha \omega t)$而
衰减,其中$t$为旅行时间深度,$\alpha$为与波动品质因数Q呈反比关系的衰减常数。在人们用这样
一种单频波来模拟真实的地震数据的时候,许多人都会误入岐途。比较好的一种模型是采用
宽频带的地震震源,例如采用脉冲函数。因为有吸收作用,高频分量衰减很迅速,最终只剩
下低频分量,因而较低频率的信号得到增强。在传播时间为$t$时,原来的白噪谱(常数谱)
为前面提到的频率之阻尼指数函数$exp(-\alpha\omega t)$所代替。形成脉冲时间函数所需的能量仅
与该频率的阻尼指数函数之下的面积是成比例的。至于说到相位,因为是假设脉冲震源而
且速度是假设为常数,所有频率成分均将是同相的(见\ref{sec:4.6}节关于因果性问题的讨论)。将
该指数函数从零至无限大频率进行积分,使我们得到负一次幂的时间$t^{-1}$,从而完成了发散
校正应为$t^2$的证明。


