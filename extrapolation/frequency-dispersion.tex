\section{频散与波场偏移精确度}
\label{sec:4.3}

频散系由不同频率成分以不同速度传播而形成。虽然许多读者可能听到过冰在冻结的湖
面和河面上滑动时的声音,可是在日常生活中却很少听说到频散物理现象。正在破裂的冰所
引起的弹性波,它的传播就具有频散性质,使得破裂声变成敲击震动的调子。地震波沿地表
面传播时一般可观察到波散,但对内部发生的反射波则简直很难察觉到有波散现象。在地震
数据处理中,频散是一种讨厌的事,对于处理的设计者来说,频散现象是严重的妨害和困
扰。频散现象主要由有限差分方法形成,因为微分算子和差分算子在高频上并不一致.进行
更加密的采样总可以压制掉频散,检查是否这样作了,正是生产分析人员的任务。图\ref{fig:dspr/freqdisp}
所示是某些经过频散的脉冲。

\begin{figure}[H]
\centering
\includegraphics[width=0.65\textwidth]{dspr/freqdisp}
\caption[freqdisp]{(a)脉冲。(b)稍微频散的脉冲,因高频耗散而形成。
(c)具有显著频散的脉冲,这类频散可以因数据处理不小心而形成}
\label{fig:dspr/freqdisp}
\end{figure}

由数据处理所引起的频散可以是关于数据正处于发
生假频的危险之中的一种很有用的警告。各种频率域方
法均不依靠差分算子,所以它们都有不会表现出频散现
象的好处,不过这种好处也有附带的后果,即(1)没
有频散只限于恒定盼物质性质和(2)没有关于频散的
警告迹象就出现了空间假频。

图\ref{fig:dspr/taner}是偏移资料中出现频散的例子,上图为共
深度点叠加剖面。中间的图是处理之后的剖面,处理时
尚未企图控制频散现象,200号炮点附近在4秒时间上存
在有严重的频散现象。下面的图则是经过重新处理之后
的剖面,频散现象已大大减弱。

\begin{figure}[H]
\centering
\includegraphics[width=0.65\textwidth]{dspr/taner}
\caption[taner]{克服和控制频散现象(据Taner与Koehler)}
\label{fig:dspr/taner}
\end{figure}

\subsection{空间假频}
\label{sec:4.3.1}

在时间轴上、深度轴上,检波点、炮点、中心点、炮检距或相交测线上,都会出现假
频。在水平空间坐标轴上,假频最为严重,\ref{sec:1.3}节的图\ref{fig:omk/alias}提供了一个例子,看这个图时,
关于是向左倾还是向右倾,你得到是混乱的概念,数学分析也览如此。波动方程的弥散关系
使我们能够利用半圆关系$k_z(\omega,k_x)=\sqrt{\omega^2/v^2-k_x^2}$从瞬时频率$\omega$,
速度$v$和水平空间波数
$k_x$计算出垂直空间波数$k_x$。在$x$坐标轴上进行采样时,令$k_x$的上限等于
Nyquist频率$\pi/\Delta x$。频率域方法和有限差分方法部要在处理高频时考虑到
在Nyquist频率上发生折叠的问题,
所以半圆形的频散关系在高于Nyguist频率时是折转的,如图\ref{fig:dspr/kxalias}所示。

\begin{figure}[H]
\centering
\includegraphics[width=0.65\textwidth]{dspr/kxalias}
\caption[kxalias]{对水平坐标采样时的波动方程有效频散关系,图中所示各
频率是按典型的零炮检距偏移给出的}
\label{fig:dspr/kxalias}
\end{figure}

当如图\ref{fig:dspr/kxalias}中20赫兹时的情形、
即两个圆彼此接触时,就开始有
空间假频问题了,这种情形出现在半
波长$v/2f$等于空间采样间隔$\Delta x$时。
爆炸反射面模型隐含承认所用速度应
为岩石速度的二分之一,因而,如果
条件$2f\Delta x < \frac{1}{2}v_{rock}$
能满足,假频问题就可避免了。就速度等于2公里/秒
的情形来说,保险的频率如下表所列:
\begin{table}[!ht]
\centering
\ttfamily
\small
\begin{tabularx}{\textwidth}{|Y|Y|Y|}
\hline
 & $\Delta x$ & 保险频率\\
\hline
标准情形& 25 m & <20 Hz\\
\hline
普查时& 50 m & <10 Hz\\
\hline
三维相交测线& 100 m & <5 Hz\\
\hline
\end{tabularx}
\end{table}

有关空间假频问题的另一种看法
是:陡倾斜低视速度的波可用检波器 组合方法压制(这种看法忽视炮点空
间假频)。根据这种观点,应该把能量从资料中消失时的射线角度考虑为是一个限度,超过
它空间假频就开始起作用。射线角度不取90°而取为30°时,水平波长增一倍,因此,对于
30°的射线角度和2公里/秒的速度,能保证不受假频影响的频率范围应如下表所列:
\begin{table}[!ht]
\centering
\ttfamily
\small
\begin{tabularx}{\textwidth}{|Y|Y|Y|}
\hline
 & $\Delta x$ & 保险频率\\
\hline
标准情形& 25 m & <40 Hz\\
\hline
普查时& 50 m & <20 Hz\\
\hline
三维相交测线& 100 m & <10 Hz\\
\hline
\end{tabularx}
\end{table}

因为资料通常有高于40赫兹的信号,所以广角
处理往往因有空间假频而遭致失败。

空间假频问题通常掩盖了
15°方程与90°方程之间的差别。受假频影响的能量并不在双曲线两翼与
顶点之间移动,往往停留原地不动。图\ref{fig:dspr/fdvrso}可说
明这种现象,该图表示由有限差分方程产生的90°
双曲线与15°双曲线。总的来说,它们差别很小。
注意双曲线初至的振幅,它们衰减得比球形扩散和
倾斜函数所预料的还要快,这是因为各波散曲线半
圆彼此重叠之故。这里可以不存在超出上限的传播
角度(超过该上限即产生沿x方向的假频现象)。由于波不可能达到如此之陡,所以它们确
实并不陡,脉冲也并非散布适当。

\begin{figure}[H]
\centering
\includegraphics[width=0.65\textwidth]{dspr/fdvrso}
\caption[fdvrso]{一对合成双曲线,$\Delta t=4$毫秒,$\Delta x=25$米,
速度为2公里/秒。采用Fourier变换方法的90°方程的双曲线(上图)及15°有限差分方程的
双曲线(下图)}
\label{fig:dspr/fdvrso}
\end{figure}

\subsection{二阶空间导数 }
\label{sec:4.3.2}

二阶差分算子的定义方程为
\begin{equation}
\frac{\delta^2}{\delta x^2}P=\frac{P(x+\Delta x)-2P(x)+P(x-\Delta x)}{(\Delta x)^2}
\label{eq:ex4.3.1}
\end{equation}
取下列极限则可定义二阶导数算子
\begin{equation}
\frac{\partial^2}{\partial x^2}P=\lim_{\Delta \to 0}\frac{\delta^2}{\delta x^2}P
\label{eq:ex4.3.2}
\end{equation}
$\Delta x$趋于零时,许多不同的定义可以全部趋于相同的极限。问题在于要求出一种表达式,该表
达式当$\Delta x$大于零时应是准确的,而且,按某种实用水平衡量,该表达式还不得过于复杂。我
们第一个目标就是要知道方程\ref{eq:ex4.3.1}的精确度如何能定量计算,第二个目标就是要考虑一
种比方程\ref{eq:ex4.3.1}稍微复杂一点但比它更精确得多的表达式。

我们将采用的基本分析方法是Fourier变换。取复指数的导数$P=P_0\exp(ikx)$,并将
任何误差均考虑为空间波数的函数,对于二阶导数,有
\begin{equation}
\frac{\partial^2}{\partial x^2}P=-k^2P
\label{eq:ex4.3.3}
\end{equation}
用类似于差分算子的表达式来定义$\hat{k}$
\begin{equation}
\frac{\delta^2}{\delta x^2}P=-\hat{k^2}P
\label{eq:ex4.3.4}
\end{equation}
理想情形是$\hat{k}$要等于k。将复指数$P_0\exp(ikx)=P$代入式\ref{eq:ex4.3.1},
可看到由定义式\ref{eq:ex4.3.4}得出以$k$表示的一种有关$k$之表达式
\begin{subequations}
\begin{equation}
-\hat{k^2}P=\frac{P_0}{(\Delta x)^2}[e^{ik(x+\Delta x)}-2e^{ikx}+e^{ik(x-\Delta x)}]
\label{eq:ex4.3.5a}
\end{equation}
\begin{equation}
-\frac{\delta^2}{\delta x^2}=\hat{k^2}=\frac{2}{(\Delta x)^2}[1-\cos(k\Delta x)]
\label{eq:ex4.3.5b}
\end{equation}
\label{eq:ex4.3.5}
根据式\ref{eq:ex4.3.5b}作出$\hat{k}\Delta x$对$k\Delta x$的关系图,是很直接了当的事。三角学的半角公式允
许我们取式\ref{eq:ex4.3.5b}的解析平方根,得
\begin{equation}
\frac{\hat{k}\Delta x}{2}=\sin\frac{k\Delta x}{2}
\label{eq:ex4.3.5c}
\end{equation}
\end{subequations}
作级数展开之后将表明,在低频情形下,$\hat{k}$确实是能很好地近似于$k$,在Nyquist频率上,
根据定义$k\Delta x=\pi$,所得近似$\hat{k}\Delta x=2$则是对于$\pi$的一种很坏的近似。


\subsection{1/6策略}
\label{sec:4.3.3}

减小$\Delta x$,恒可提高绝对精度。增加较高阶项,例如
\begin{equation}
\frac{\partial^2}{\partial x^2}\approx\frac{\delta^2}{\delta x^2}
-\frac{\Delta x^2}{12}\frac{\delta^4}{\delta x^4}+\text{等等}
\label{eq:ex4.3.6}
\end{equation}
则需要以增加计算机时间和分析时的麻烦为代价,换得提高与Nyquist频率有关的精度。当
$\Delta x$趋于零时,方程\ref{eq:ex4.3.6}趋于基本定义式\ref{eq:ex4.3.1}和\ref{eq:ex4.3.2}。
如果希望在$k$值小时有
高精度,可以用Taylor级数方法决定式\ref{eq:ex4.3.6}中像1/12这类的系数。或者,如果希望在
办值的某种范围内有一定精度,可以用曲线拟合方法来确定稍微有点不同的系数。在实际处
理中,极少采用式\ref{eq:ex4.3.6},因为还有一种不太明显的表达式能以很少代价就可提供高得
多的精度!这种思想可用下式简要说明:
\begin{subequations}
\begin{equation}
\frac{\partial^2}{\partial x^2}\approx
\frac{\frac{\delta^2}{\delta x^2}}{1+b\Delta x^2\frac{\delta^2}{\delta x^2}}
\label{eq:ex4.3.7a}
\end{equation}
式中,$b$是一个可调常数。
将式\ref{eq:ex4.3.5b}代入,得
\begin{equation}
(\frac{\hat{k}\Delta x}{2})^2=
\frac{\sin^2(\frac{k\Delta x}{2})}{1-b4\sin^2(\frac{k\Delta x}{2})}
\label{eq:ex4.3.7b}
\end{equation}
\label{eq:ex4.3.7}
\end{subequations}
就能对式\ref{eq:ex4.3.7a}的精度作出数值估计。在取值$b=
1/6$的情形下,对上式取平方根的计算结果绘于图\ref{fig:dspr/sixth}。

\begin{figure}[H]
\centering
\includegraphics[width=0.65\textwidth]{dspr/sixth}
\caption[sixth]{作为空间波数之函数的二阶导数表示式\ref{eq:ex4.3.7}的精度
式\ref{eq:ex4.3.7b}之平方根的符号在$-\pi$至$\pi$满范围内取成与$k$一致,而在该范围
之外则为周期性质(据Hale)}
\label{fig:dspr/sixth}
\end{figure}

% 图\ref{fig:dspr/undermig}中的误差全部是某种时移误差。由于沿反射面的反射系数为常数,识别不出什
% 么横向位移误差。该时间误差在理论上可按下式确定:
% \begin{equation}
% \frac{dt}{t}\approx\frac{dz}{z}\approx\frac{\hat{k_z}-k_z}{k_z}
% \label{eq.ex4.2.2}
% \end{equation}
% 就所谓15°外推方程来说,在25°度时所造成的相位误差可证明大约为百分之五十。

% 其次,我们要确定一下双曲线收缩压扁时的误差。图\ref{fig:dspr/errcollapse}所示是一个双曲线的向下延
% 拓,为清晰起见,向下延拓没有沿指向聚焦点的所有路径进行。选定某个斜率$p$就可选出具
% 有某个Snell参量$p=dt/dx$的射线。试想像有一斜率为$p$的切线段切于各双曲线。如果斜率
% 为p之处有一点振幅异常。那末,你就能在各个双曲面上追踪出它。

% \begin{figure}[H]
% \centering
% \includegraphics[width=0.65\textwidth]{dspr/errcollapse}
% \caption[errcollapse]{双曲线挤缩收敛误差。
% 注意,实际曲线位于所期望曲线之上,但是
% 实际的点却是位于所期望点之下}
% \label{fig:dspr/errcollapse}
% \end{figure}

% 在图\ref{fig:dspr/errcollapse}中,时间移动量太小了,横向移动距离同样也很小。实际上,具有$r_0=1$的
% 15°方程的误差,有时采取把深度坐标$z$或者速度$v$增大6\%的办法就能被补偿掉。各误差量可
% 根据下式计算
% \begin{subequations}
% \begin{equation}
% \frac{\Delta t}{t}=\frac{\frac{\partial}{\partial \omega}(\hat{k_z}-k_z)}{\frac{\partial}{\partial \omega}k_z}
% \label{eq:ex4.2.3a}
% \end{equation}
% \begin{equation}
% \frac{\Delta x}{x}=\frac{\frac{\partial}{\partial k_x}(\hat{k_z}-k_z)}{\frac{\partial}{\partial k_x}k_z}
% \label{eq:ex4.2.3b}
% \end{equation}
% \label{eq:ex4.2.3}
% \end{subequations}
% 式中,$k_z$取为$\omega$和$k_x$的函数。它证明,就15°外推方程而言,在角度为20°度时出现百分之五
% 十左右的群速度误差,因而群速度误差一般是要比相速度误差更为严重的。

% \subsection{群速度方程的导出}
% \label{sec:4.2.4}
% 在$(x,z)$空间的原点上之脉冲函数是许多Fourier分量的叠加结果:
% \begin{equation}
% \iint e^{+ik_xx+ik_zz}dk_xdk_z
% \label{eq:ex4.2.4}
% \end{equation}
% 根据物理规律(或许还有根据数值分析)所导出的弥散关系是$\omega$、$k_x$与$k_z$之间的一种函数关
% 系,比方说,以$\omega(k_x,k_z)$表示这种关系。由标量波动方程得出的就是这
% 一类弥散关系的最普通的例子。该方程的解为
% \begin{equation}
% e^{-i\omega t+ k_xx+ik_zz}
% \label{eq:ex4.2.5}
% \end{equation}
% 遍及$(k_x,k_z)$对式\ref{eq:ex4.2.5}进行积分,得出一种单频时间函数,该函数在$t=0$时就是
% 位于$(x,z)=(0,0)$点上的脉冲。在某个非常大的时间$t$时,这种函数的表达式为
% \begin{equation}
% \iint e^{-it[\omega(k_x,k_z)-k_xx/t-k_zz/t]}dk_xdk_z
% \label{eq:ex4.2.6}
% \end{equation}
% 在$t$非常之大时,被积函数是具有单位幅度的非常快速振荡之函数。因此,除非知道方括号
% 内的量在$(k_x,k_z)$空间相当广大的区域内近乎与$k_x$和$k_z$独立无关,否则这个积分将接近于
% 零。像求出一个二维函数之极大值和极小值的作法一样,就是说,采用令其各个偏导数等于
% 零的办法,就可以求出这样一种平点(flat spot)。这种分析方法即是众所围知的稳相法
% (stationary phase method)。由它得出
% \begin{subequations}
% \begin{equation}
% \frac{\partial}{\partial k_x}[]=\frac{\partial\omega}{\partial k_x}-\frac{x}{t}=0
% \label{eq:ex4.2.7a}
% \end{equation}
% \begin{equation}
% \frac{\partial}{\partial k_z}[]=\frac{\partial\omega}{\partial k_z}-\frac{z}{t}=0
% \label{eq:ex4.2.7b}
% \end{equation}
% \label{eq:ex4.2.7}
% \end{subequations}
% 所以,最终得出结论,在时间$t$时, 该扰动将位于
% \begin{equation}
% (x,z)=t(\frac{\partial\omega}{\partial k_x},\frac{\partial\omega}{\partial k_z})
% \label{eq:ex4.2.8}
% \end{equation}
% 由此证明群速度的定义是正确的。
% 现在让我们看一看图\ref{fig:dspr/anisoxz}中的15°外推方程情形下的波阵面是如何计算出来的。解出
% 15°弥散关系的$\omega$值并代入式\ref{eq:ex4.2.8},所得结果$(x,z)$证明是的一个函数,利用一
% 切可能的值就可绘出该图中的曲线。

% \subsection{能量偏移方程的导出}
% \label{sec:4.2.5}

% 现在以类似于导出群速度的方式来分析$(x,t)$空间内的能量偏移。将积分
% \begin{equation}
% \iint e^{iz[k_z(\omega,k_x)-\omega t/z+k_xx/2]}d\omega dk_x
% \label{eq:ex4.2.9}
% \end{equation}
% 内的深度z取得很大,结果就使能量趋向下列位置
% \begin{equation}
% (x,t)=z(-\frac{\partial k_z}{\partial k_x},\frac{\partial k_z}{\partial \omega})
% \label{eq:ex4.2.10}
% \end{equation}
% 这就证明了我们以前断言可以用式\ref{eq:ex4.2.3}来分析能量传播误差一事是正确的。式
% \ref{eq:ex4.2.10}也可用于计算图\ref{fig:dspr/anisoxt}中的曲线。稳相概念的有效性已为图
% \ref{fig:dspr/45imp}所证实,该图就是利
% 用反Fourier变换方法得到的。
% %
% \subsection{外推方程不具有频散性质}
% \label{sec:4.2.6}

% 为证明所熟悉的15°、45°等等的波场外推方法不是频散的,回想一下\ref{sec:2.2}节,在该节
% 中,弥散关系都具有$k_z/\omega=f(k_x/\omega)$
% 的形式,其中,$f$是某种半圆近似,比方说,15°近似
% 或45°近似。这种形式的弥散关系没有一种能够是频散的。作出式(4.2.10)所要求的求导运
% 算,你就会明白,虽然波阵面的$(x,t)$坐标通过Snell参量$vk_x/\omega$而与倾角有关,可它们并非明
% 显迪直接依赖于频率$\omega$。所以,实践中所观察到的任何频散并不是由15°或45°近似所产生的。
