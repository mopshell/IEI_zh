\section{阻抗}
\label{sec:4.6}

经典物理学非常注意能量守恒与耗散问题,工程滤波理论非常注意时间因果性问题、即
激发之前不可能存在响应。在地球物理学中,我们经常需要既能保证时间因果性关系,又要
考虑能量损耗的问题。我们需要不但是在理论推导方面而且在计算方面将二者结合起来,有
时在时间离散化的计算中也需如此。有专门一类称怍阻抗函数的数学函数可用以描述在耗散
能量之物理物体中的因果性线性扰动。

大自然是按时间向前推移发展的。很自然,阻抗函数在任何模拟计算中都是起基本作用
的,在那里,时间是从过去向未来推移发展的。除了它们在物理模拟中的应用以外,阻抗在波
场的深度外推中也发现有用处。地球物理学家将地表面上获得的数据向下外推,以期获得深
处的信息,这种作法不同于大自然沿时间的外推。在原理上,我们并不要求将阻抗函数向深
处外推,但是进行深度偏移而不考虑阻抗函数却会出现不断増大的振荡现象,这一点同接受
外部来源能量的物理系统非常相像。事实上,物理方程的“直接”实现往往表现为不稳定的
外推,若用阻抗函数来表示我们的外推问题,那我们就能保证稳定性。在一种算法所能具有
的一切长处,如稳定性、精度、清晰性、普遍性、效率、模块性等等之中,最为重要的看来
就属稳定性了。

在本节中,我们要研究阻抗函数理论,研究其精确定义、及其离散时间域内的计算方
法、以及由简单阻抗组合成更复杂阻抗函数的规律;我们还要研究其他一些特殊函数、如极
小相位滤波与反射率滤波(reflectance filter)等与阻抗滤波的关系。时间域内的广角波场
外推与偏移将以阻抗函数来表示。岩石因其含有一切尺度的不规则性而与“纯”物质不相
像,我们将求出一种特别简单的阻抗函数来模仿岩石内的能量耗散,它不同于经典的牛顿粘
滞性方程。

\subsection{谨防得出无穷大结果}
\label{sec:4.6.1}

取一个无穷级数,如1,
-1、+1、-1、+1$\ldots\ldots$,以两种不同方式将级数各读组合并
按下列方式将它们相加,可以得出1等于0的荒谬结果:
\begin{equation*}
(1-1)+(1-1)+(1-1)+...=1+(-1+1)+(-1+1)+...
\end{equation*}
\begin{equation*}
0+0+0+...=1+0+0+...
\end{equation*}
当然,这并非真正证明一等于零,而只是证明对无穷级数必须很小心地处理而已。其次,再
取另一种无穷级数,其中各项应可按任意顺序重新分类组合而无需担心得出似是而非的结
果.要得到这类级数,可这么设想:把一个馅饼一分为二,将其中一半再一分为二,得到两
个四分之一的饼,然后再将其中一个四分之一分成两个八分之一的饼$\ldots$,就如此进行下去。
这样的无穷级数就是1/2、1/4、1/8、1/16......,不论把已被分为小块的焰饼如何重新排
列,它们应该总能放回到盛馅饼的盘子内,并准确地把它填满。

无穷级数的危险并不在于它们有无限多项,而在于将它们求和可得出无穷大结果。如果
各项绝对值之和为有限,那就能保证没事,这样一类级数称作绝对收敛级数。

\subsection{Z变换}
\label{sec:4.6.2}


任意离散时间函数$x_t$之Z变换定义为
\begin{equation}
X(Z)=......x_{-2}Z^{-2}+x_{-1}Z^{-1}+x_0+x_1Z+x_2Z^2
\label{eq:ex4.6.1}
\end{equation}
在物理上可把变量Z解释为有一个单位时间的时延,于是之$Z^2$就代表有两个单位时间的延迟,
等等。像$X(Z)U(Z)$和$X(Z)U(1/Z)$这样的表达式在以后是很有用处的,因为它们
的意思是指时间域系数的褶积和互相关(见《地球物理数据处理基础》一书)。

再往下考虑延迟算子$Z$的数值,我们发现,很有用的是问一下$X(Z)$究竟是有限的还是
无限的。算子$Z$具有特殊意义的数值是$Z=+1$和$Z=-1$,以及$Z$具有单位幅度$\mid Z\mid =1$
的所有复数值,或即
\begin{equation}
Z=e^{i\omega \Delta t}
\label{eq:ex4.6.2}
\end{equation}

式中,$\omega$为Fourier变换的实变量。取$\omega$为实数意味着Z是在单位圆上,这时的Z变换是一种
离散Fourier变换。由于我们要求$U(Z)$对单位圆$\mid Z\mid=1$上的所有Z值均为有限,我们可将
注意力局限于具有有限能量的时间函数。滤波函数总是局限于具有有限能量的。

说滤波是具有因果性,其最直接了当的说法就是说它的时间域系数在零时间以前均等于
零,亦即对于$t<0$应有$u_t=0$;另一种说法就是说,对于$Z=0$,$U(Z)$
应有限。如若各系数
$u_{-1}$、$u_{-2}$...等等不为零,则在Z=0时该$Z$变换将为无限大。对于一个具有因果性的函数,当
Z取在单位圆内$\mid Z\mid<1$而不是取在单位圆$\mid Z\mid=1$上时,$\mid U(Z)\mid$的各项均将比较小,所以
在$Z=0$时和在单位圆$\mid Z\mid=1$上时具有收敛性就是暗示在单位圆内应处处收敛。因此,将有
界性同因果性结合起来就是意味着在单位圆内收敛。在$Z=0$时收敛而在单位圆上$\mid Z\mid=1$不
收敛的情形,属于具有无限能量的因果性函数,这种情形没有实用意义。什么种类的函数是
在单位圆上和在$Z=\infty$时收敛而在$Z=0$时不收敛?什么函数是在所有Z=0、$Z=\infty$和$\mid Z\mid=1$
这三种情形下均收敛?

滤波算子$1/(1-2Z)$至少可按两种不同方式展为Z的幂级数,这两种方式为:
\begin{equation}
\frac{1}{1-2Z}=1+2Z+4Z^2+8Z^3+...\\
=-\frac{1}{2Z}\frac{1}{1-\frac{1}{2Z}}=\frac{-1}{2Z}[1+\frac{1}{2Z}+\frac{1}{4Z^2}+......]
\label{eq:4.6.3}
\end{equation}
这两种无穷级数中哪一种的收敛是与Z的数值有关?对于$\mid Z\mid=1$来说,第一种级数发散,但
第二种级数却收敛,所以,仅有反因果性滤波才是可接受的滤波。级数展开
是唯一的吗?如果收敛,它就是唯一的,复变函数理论可以证明这个结论。

设以$b_t$表示某一滤波器,如果$b_t$与$a_t$的褶积结果是$\delta$函数,则$a_t$就是$b_t$的反滤波器,在
Fourier域内我们就说,如其Fourier变换彼此可逆,则两个滤波器就彼此可逆。反滤波可用
Z变换定义,例如$A(Z) =1/B(Z)$。滤波$A(Z)$是否具有因果性就看它是否在单位圆内
处处有限,或者说,实际是与$B(Z)$是否在单位圆内任何处均等于零有关。例如$B(Z)=1-2Z$在$Z=1/2$时为零,此时$A(Z)=1/B(Z)$必然为无穷大,就是说,级数$A(Z)$在
$Z=l/2$时必然不收敛,因而$a_t$不具有因果性。当滤波$B(Z)$及其倒数均具因果性时,出现
一种最有意义的情形,称为极小相位,以上所述可总结如下:
\begin{table}[!ht]
\centering
\ttfamily
\small
\begin{tabularx}{\textwidth}{Y|Y}
\hline
因果性& $\mid B(Z)\mid<\infty 当\mid Z\mid \le 1$\\
\hline
因果性倒数& $\mid 1/B(Z)\mid<\infty 当\mid Z\mid \le 1$\\
\hline
极小相位 &满足上述两项条件 \\
\hline
\end{tabularx}
\end{table}

\subsection{阻抗滤波器评述}
\label{sec:4.6.3}

利用Z变换的记号定义一种滤波$R(Z)$,其输入为$X(Z)$,其输出为$Y(Z)$,于是
\begin{equation}
Y(Z)=R(Z)X(Z)
\label{eq:ex4.6.4}
\end{equation}
如$R(Z)$的级数形式表示式没有Z的负幂项,就说该滤波是因果性滤波。换句话说,$y_t$可根
据$x_t$的过去值与现在值确定。再者,如$l/R(Z)$不含Z的负幂项,则该滤波$R(Z)$为极小
相位的,这意味着,采用直接的多项式除法
\begin{equation}
X(Z)=\frac{Y(Z)}{R(Z)}
\label{eq:4.6.5}
\end{equation}
即可根据$y_t$的现在值与过去值决定$x_t$。

设$R(Z)$已经是极小相位,若正值的能量或功可表示如下:
\begin{subequations}
\begin{equation}
0\le \text{功}=\sum_t \text{力}\times\text{速度}=\sum_t\text{电压}\times\text{电流}
\label{eq:ex4.6.6a}
\end{equation}
\begin{equation}
=\frac{1}{2}\sum_t(\bar{x_t}y_t+\bar{y_t}x_t)
\label{eq:ex4.6.6b}
\end{equation}
\begin{equation}
=[\bar{X}(\frac{1}{Z})Y(Z)+\bar{Y}(\frac{1}{Z})X(Z)]\text{的}Z^0\text{项之系数}
\label{eq:ex4.6.6c}
\end{equation}
\begin{equation}
=\frac{1}{2\pi}\int_0^{2\pi}Re(\bar{X}Y)d\omega
\label{eq:ex4.6.6d}
\end{equation}
\begin{equation}
=\int Re(\bar{X}RX)d\omega =\int \bar{X}X Re(R)d\omega
\label{eq:ex4.6.6e}
\end{equation}
\end{subequations}
则$R(Z)$还可以是一种阻抗函数。既然$\bar{X}X$可以是位于任何频率$\omega$
上的脉冲函数,从而可以得结论:$Re[R(\omega)]\ge 0$对于所有实数的
$\omega$均成立。以上所述可总结如下:
\begin{table}[!ht]
\centering
\ttfamily
\small
\begin{tabularx}{\textwidth}{Y|Y}
\hline
因果性& $r_t=0$,对于$t<0$即$\mid R(Z)\mid<\infty$,对于$\mid Z\mid \le 1$\\
\hline
因果性倒数& $\mid 1/R(Z)\mid<\infty 当\mid Z\mid \le 1$\\
\hline
耗散能量 &$2ReR(\omega)=R(Z)+\bar{R}(1/Z)\geq 0$,$\omega$为实数 \\
\hline
\end{tabularx}
\end{table}


将阻抗函数与其Fourier共轭函数相加,得出一种像功率谱一般的实数正依函数(其虚
部等于零),例如
\begin{subequations}
\begin{equation}
(r_0+r_1Z+r_2Z^2+...)+(\bar{r_0}+\bar{r_1}\frac{1}{Z}+\bar{r_2}\frac{1}{Z^2}
+...) \geq 0, \text{对实数$\omega$}
\label{eq:ex4.6.7a}
\end{equation}
\begin{equation}
R(Z)+\bar{R(\frac{1}{Z}}\geq 0, \text{对实数$\omega$}
\label{eq:ex4.6.7b}
\end{equation}
\label{eq:ex4.6.7}
\end{subequations}

\subsection{因果性积分}
\label{sec:4.6.4}

设有离散时间函数其Fourier变换经代换$Z=exp(i\omega \Delta t)$之后,得Z变换为
\begin{equation}
P(z)=......+P_{-2}Z^{-2}+p_{-1}Z^{-1}+p_0+p_1Z+p_2Z^2+......
\label{eq:ex4.6.8}
\end{equation}
根据以下的关系,定义一个算子$-i\hat{\omega}$
\begin{equation}
\frac{1}{-i\hat{\omega}\Delta t}=\frac{1}{2}\frac{1+Z}{1-Z}
\label{eq:ex4.6.9}
\end{equation}
将此算子应用于$P(Z)$,定义出另一个离散时间函数$q_t$,其Z变换为Q(Z)
\begin{equation}
Q(Z)=\frac{1}{2}\frac{1+Z}{1-Z}P(Z)
\label{eq:ex4.6.10}
\end{equation}
两端乘以$(1-Z)$,得
\begin{equation}
(1-Z)Q(Z)=\frac{1}{2}(1+Z)P(Z)
\label{eq:ex4.6.11}
\end{equation}
令两端同幂项$Z^t$的系数相等,则得
\begin{equation}
q_t-q_{t-1}=\frac{p_t+p_{t-1}}{2}
\label{eq:ex4.6.12}
\end{equation}
令$p_t$为脉冲函数,我们就可看出$q_t$原来是一种阶跃函数,即
\begin{subequations}
\begin{equation}
p=......,0,0,0,0,0,1,0,0,0,0,0,......
\label{eq:ex4.6.13a}
\end{equation}
\begin{equation}
q=......,0,0,0,0,0,\frac{1}{2},1,1,1,1,1,......
\label{eq:ex4.6.13b}
\end{equation}
\label{eq:ex4.6.13}
\end{subequations}

% \subsection{Stolt拉伸扩展法}
% \label{sec:4.5.4}
%
% 采用深度方向的Fourier变换是Stolt偏移方法的一大长处,同时也是它的一大弱点。这是
% 一大长处是因为这使他的方法在运算上比所有其他方法都快速得多,这又是一个弱点是因为
% 它要求速度必须是深度的一种恒定函数。在地震剖面范围之内变化两倍是速度的典型变化范
% 围,而速度对偏移的影响则往往与速度的平方有关。为减轻这种困难,Stoit曾建议将时间
% 坐标轴拉伸扩展,使得数据看上去很像是由某一恒定速度的地层所产生的。
% Stolt提出过下 列拉伸扩展函数:
% \begin{subequations}
% \begin{equation}
% \tau(t)=(\frac{2}{v_0^2}\int_0^t t'v_{RMS}^2(t')dt')^{1/2}
% \label{eq:ex4.5.1a}
% \end{equation}
% \begin{equation}
% v_{RMS}^2(t)=\frac{1}{t}\int_0^t v^2(t')dt'
% \label{eq:ex4.5.1b}
% \end{equation}
% \label{eq:ex4.5.1}
% \end{subequations}
%
% 在涉及高速度的很大时间之处,Stolt的拉伸扩展意味着$\tau$值的增长要比t值快速一些。为能
% 应用快速傅氏变换FFT,要沿$\tau$轴均匀采样,这么一来,在很大时间之处,就得在t坐标轴上增大
% 采样密度;这正同关于地层Q值和采样定理的要求相反,但是大多数人都认为值得这么作。
%
% 导出式\ref{eq:ex4.5.1}的最直接方法是基于这么一种思想:要使理想双曲线顶部的曲率同拉
% 伸扩展之数据中的曲率相符一致。$(x,\tau)$ 空间内的理想双曲线方程为
% \begin{equation}
% v_0^2\tau^2=x^2+z^2
% \label{eq:ex4.5.2}
% \end{equation}
% 作简单的微分即可表明,双曲线顶部的曲率为
% \begin{equation}
% \frac{d^2\tau}{dx^2}\mid_{x=0}=\frac{1}{\tau v_0^2}
% \label{eq:ex4.5.3}
% \end{equation}
% 可以证明,除了速度必须用均方根速度代替之外
% \begin{equation}
% \frac{d^2 t}{dx^2}\mid_{x=0}=\frac{1}{tv_{RMS}^2}
% \label{eq:ex4.5.4}
% \end{equation}
% 式\ref{eq:ex4.5.3}是可应用于层状介质内的。我们现在要针对层状介质寻找出一种经拉伸扩展的
% 时间$\tau(t)$,使得可以用式\ref{eq:ex4.5.3}有效地代替式\ref{eq:ex4.5.4}。我们很想使$t(x)$曲线与
% $(x)$曲线在所有$x$值情形下都能匹配,但那就会使问题过于复杂化了。退而求其次,我们
% 可以只求在双曲线顶部的导数能相符、即与$x=0$点上的$\tau[t(x)]$对$x$之二阶导数能够匹
% 配。代入式\ref{eq:ex4.5.3}与式\ref{eq:ex4.5.4}的关系,可由这种作法导出$\tau d\tau/dt$而的表达式。在进行
% 积分并取其平方根之后,即得出式\ref{eq:ex4.5.1}。
%
% 有一种与此不同的导出拉伸扩展时间$\tau$的方法,它可以在较陡角度时给出更精确的结
% 果,这种方法不是在顶点上使双曲线曲率匹配,而是在侧翼上离开顶部有若干距离之处使双
% 曲线斜率和函数值均匹配。实际偏移的正是双曲线的两翼而不是顶部,所以这种结果就要更
% 准确些。从代数运算说,这种导出方法也较容易,因为所需要的仅是一阶导数。对于任意深
% 度$z_i$上的反射面,将式\ref{eq:ex4.5.2}对$x$微分,得出
% \begin{equation}
% \frac{d\tau}{dx}=\frac{x}{\tau v_0^2}
% \label{eq:ex4.5.5}
% \end{equation}
% 在层状介质内也存在类似的表达式,为得出这个表达式,解出$x=\int v\sin\theta dt=p\int v^2 dt$,
% 就得出
% $p=dt/dx$为:
% \begin{equation}
% \frac{dt}{dx}=\frac{x}{\int_0^tv^2(p,t)dt}
% \label{eq:ex4.5.6}
% \end{equation}
% 表达式\ref{eq:ex4.5.5}与\ref{eq:ex4.5.6}起着与式\ref{eq:ex4.5.3}和式\ref{eq:ex4.5.4}相同的作用,但式\ref{eq:ex4.5.5}与式\ref{eq:ex4.5.6}在任何地方均成立,而不仅只在双曲线顶部成立。对$\tau(t)$进行微分,得
% \begin{equation}
% \frac{d\tau}{dx}=\frac{d\tau}{dt}\frac{dt}{dx}
% \label{eq:ex4.5.7}
% \end{equation}
% 将式\ref{eq:ex4.5.5}与\ref{eq:ex4.5.6}代入式\ref{eq:ex4.5.7},得
% \begin{equation}
% \frac{x}{\tau v_0^2}=\frac{d\tau}{dt}\frac{x}{\int_0^tv^2(p,t)dt}
% \label{eq:ex4.5.8}
% \end{equation}
% \begin{equation}
% \tau d\tau=[\frac{1}{v_0^2}\int_0^tv^2(p,t')dt']dt
% \label{eq:ex4.5.9}
% \end{equation}
% 对式\ref{eq:ex4.5.9}进行积分,其左端得出$\tau^2/2$,这时取平方根即可得出式
% \ref{eq:ex4.5.1a},但须采用下列均方根速度$v_{RMS}$的新定义
% \begin{equation}
% v_{RMS}^2=\frac{1}{t}\int_0^tv^2(p,t)dt
% \label{eq:ex4.5.1c}
% \end{equation}
%
% 现在的新鲜事是出现了Snell参量p。在以某种速度$v'(z)$为其特征的一种层状介质
% 内,$v(p,t)$定义为射线速度,该射线对地面呈某个角度,可用时差p量度。在实际处理
% 中,应该采用什么p值?最好的处理办法是先检查一下所用资料,针对你希望对它进行良好
% 偏移的那些同相轴测定出,$p=dt/dx$;有一个缺省值为
% \begin{equation*}
% p=\frac{2\sin30^{\circ}}{2.5km/s}=0.4ms/m
% \end{equation*}
% 式中的因子2是根据爆炸反射面模型得出的。
%
% \subsection{Gazdag的横向可变速度$v(x)$处理法}
% \label{sec:4.5.5}
%
% 相移偏移方法很受欢迎,因为它允许速度随深度作任意变化,而且允许传播角度任意,
% 直至为90°。可惜,由于沿x坐标作Fourier变换,不允许速度有横向变化。为减轻这种困
% 难,JenoGazdag与Sguazzero(l984)提出过一种内插方法。回想一下\ref{sec:1.3}节内容,相移法
% 把数据$p(x,t)$通过二维Fourier变换,转换为$P(k_x,\omega)$,然后,乘以$\exp[ik_z(\omega,k_x)\Delta z]$
% 即可将$P(k_x,\omega)$按深度步长向下延拓。
% Gazdag提出了若干个参考速度$v_1$、$v_2$、
% $v_3$和$v_4$等,他用每一种速度向下延拓一个深度步长,获得经过向下延拓处理之数据的若干参
% 考结果$P_1$、$P_2$、$P_3$和$P_4$等,然后按横向波数$k_x$将每一种$P_i$作Fourier反变换,转换为
% $p_i(x,\omega)$。在每一种x值时,他都对速度最相近的那些参考波进行了内插,得出最终值
% $p(x,\omega)$。为下一个步长作准备,他把$p(x,\omega)$重又变换成$P(k_x,\omega)$,重复前面的
% 处理过程,就这样逐个步长作下去。由于对每一种参考速度情形都是重复通常的偏移计算,
% 这看来好像是一种效率不高的方法。使人惊奇的是,也许因为有利用阵列处理机进行计算所
% 具有的独特性质,这种方法似乎是很成功的。
%
% \subsection{习 题}
% \label{sec:4.5.6}
%
% \begin{enumerate}
% \item 为在时间$t_c$获得尖锐的截止,就需要频率域内的谱具有宽广频带。设已知图\ref{fig:crft/hyptrunc}是在一种$1000\times
% 1000$的网格上表示的,试推断出在截止时间$t^c$内的不确定度(uncertainty)。
% \item 因为对时间t的Fourier变换具有周期性,相移方法往往产生对$z$轴是周期性的偏移。通
% 常,这并不引起麻烦,因为我们不理会很大的$z$很大深度上的上行波在$t=0$
% 以前应该是等
% 于零的。Kjartansson曾指出,如果在继续往下进行计算之前,从波场中减掉在$t=0$时的
% 波,那就可以避免沿深度$z$的周期性,因而信息永不可能与负时间和“折叠”现象沾边。试
% 指出应如何改变相移方法的程序。
% \end{enumerate}
%
